\documentclass[]{article}
\usepackage{lmodern}
\usepackage{amssymb,amsmath}
\usepackage{ifxetex,ifluatex}
\usepackage{fixltx2e} % provides \textsubscript
\ifnum 0\ifxetex 1\fi\ifluatex 1\fi=0 % if pdftex
  \usepackage[T1]{fontenc}
  \usepackage[utf8]{inputenc}
\else % if luatex or xelatex
  \ifxetex
    \usepackage{mathspec}
  \else
    \usepackage{fontspec}
  \fi
  \defaultfontfeatures{Ligatures=TeX,Scale=MatchLowercase}
\fi
% use upquote if available, for straight quotes in verbatim environments
\IfFileExists{upquote.sty}{\usepackage{upquote}}{}
% use microtype if available
\IfFileExists{microtype.sty}{%
\usepackage{microtype}
\UseMicrotypeSet[protrusion]{basicmath} % disable protrusion for tt fonts
}{}
\usepackage[margin=2.54cm]{geometry}
\usepackage{hyperref}
\hypersetup{unicode=true,
            pdftitle={Assignment 5: Water Quality in Lakes},
            pdfauthor={Lindsay Roth},
            pdfborder={0 0 0},
            breaklinks=true}
\urlstyle{same}  % don't use monospace font for urls
\usepackage{color}
\usepackage{fancyvrb}
\newcommand{\VerbBar}{|}
\newcommand{\VERB}{\Verb[commandchars=\\\{\}]}
\DefineVerbatimEnvironment{Highlighting}{Verbatim}{commandchars=\\\{\}}
% Add ',fontsize=\small' for more characters per line
\usepackage{framed}
\definecolor{shadecolor}{RGB}{248,248,248}
\newenvironment{Shaded}{\begin{snugshade}}{\end{snugshade}}
\newcommand{\AlertTok}[1]{\textcolor[rgb]{0.94,0.16,0.16}{#1}}
\newcommand{\AnnotationTok}[1]{\textcolor[rgb]{0.56,0.35,0.01}{\textbf{\textit{#1}}}}
\newcommand{\AttributeTok}[1]{\textcolor[rgb]{0.77,0.63,0.00}{#1}}
\newcommand{\BaseNTok}[1]{\textcolor[rgb]{0.00,0.00,0.81}{#1}}
\newcommand{\BuiltInTok}[1]{#1}
\newcommand{\CharTok}[1]{\textcolor[rgb]{0.31,0.60,0.02}{#1}}
\newcommand{\CommentTok}[1]{\textcolor[rgb]{0.56,0.35,0.01}{\textit{#1}}}
\newcommand{\CommentVarTok}[1]{\textcolor[rgb]{0.56,0.35,0.01}{\textbf{\textit{#1}}}}
\newcommand{\ConstantTok}[1]{\textcolor[rgb]{0.00,0.00,0.00}{#1}}
\newcommand{\ControlFlowTok}[1]{\textcolor[rgb]{0.13,0.29,0.53}{\textbf{#1}}}
\newcommand{\DataTypeTok}[1]{\textcolor[rgb]{0.13,0.29,0.53}{#1}}
\newcommand{\DecValTok}[1]{\textcolor[rgb]{0.00,0.00,0.81}{#1}}
\newcommand{\DocumentationTok}[1]{\textcolor[rgb]{0.56,0.35,0.01}{\textbf{\textit{#1}}}}
\newcommand{\ErrorTok}[1]{\textcolor[rgb]{0.64,0.00,0.00}{\textbf{#1}}}
\newcommand{\ExtensionTok}[1]{#1}
\newcommand{\FloatTok}[1]{\textcolor[rgb]{0.00,0.00,0.81}{#1}}
\newcommand{\FunctionTok}[1]{\textcolor[rgb]{0.00,0.00,0.00}{#1}}
\newcommand{\ImportTok}[1]{#1}
\newcommand{\InformationTok}[1]{\textcolor[rgb]{0.56,0.35,0.01}{\textbf{\textit{#1}}}}
\newcommand{\KeywordTok}[1]{\textcolor[rgb]{0.13,0.29,0.53}{\textbf{#1}}}
\newcommand{\NormalTok}[1]{#1}
\newcommand{\OperatorTok}[1]{\textcolor[rgb]{0.81,0.36,0.00}{\textbf{#1}}}
\newcommand{\OtherTok}[1]{\textcolor[rgb]{0.56,0.35,0.01}{#1}}
\newcommand{\PreprocessorTok}[1]{\textcolor[rgb]{0.56,0.35,0.01}{\textit{#1}}}
\newcommand{\RegionMarkerTok}[1]{#1}
\newcommand{\SpecialCharTok}[1]{\textcolor[rgb]{0.00,0.00,0.00}{#1}}
\newcommand{\SpecialStringTok}[1]{\textcolor[rgb]{0.31,0.60,0.02}{#1}}
\newcommand{\StringTok}[1]{\textcolor[rgb]{0.31,0.60,0.02}{#1}}
\newcommand{\VariableTok}[1]{\textcolor[rgb]{0.00,0.00,0.00}{#1}}
\newcommand{\VerbatimStringTok}[1]{\textcolor[rgb]{0.31,0.60,0.02}{#1}}
\newcommand{\WarningTok}[1]{\textcolor[rgb]{0.56,0.35,0.01}{\textbf{\textit{#1}}}}
\usepackage{graphicx,grffile}
\makeatletter
\def\maxwidth{\ifdim\Gin@nat@width>\linewidth\linewidth\else\Gin@nat@width\fi}
\def\maxheight{\ifdim\Gin@nat@height>\textheight\textheight\else\Gin@nat@height\fi}
\makeatother
% Scale images if necessary, so that they will not overflow the page
% margins by default, and it is still possible to overwrite the defaults
% using explicit options in \includegraphics[width, height, ...]{}
\setkeys{Gin}{width=\maxwidth,height=\maxheight,keepaspectratio}
\IfFileExists{parskip.sty}{%
\usepackage{parskip}
}{% else
\setlength{\parindent}{0pt}
\setlength{\parskip}{6pt plus 2pt minus 1pt}
}
\setlength{\emergencystretch}{3em}  % prevent overfull lines
\providecommand{\tightlist}{%
  \setlength{\itemsep}{0pt}\setlength{\parskip}{0pt}}
\setcounter{secnumdepth}{0}
% Redefines (sub)paragraphs to behave more like sections
\ifx\paragraph\undefined\else
\let\oldparagraph\paragraph
\renewcommand{\paragraph}[1]{\oldparagraph{#1}\mbox{}}
\fi
\ifx\subparagraph\undefined\else
\let\oldsubparagraph\subparagraph
\renewcommand{\subparagraph}[1]{\oldsubparagraph{#1}\mbox{}}
\fi

%%% Use protect on footnotes to avoid problems with footnotes in titles
\let\rmarkdownfootnote\footnote%
\def\footnote{\protect\rmarkdownfootnote}

%%% Change title format to be more compact
\usepackage{titling}

% Create subtitle command for use in maketitle
\providecommand{\subtitle}[1]{
  \posttitle{
    \begin{center}\large#1\end{center}
    }
}

\setlength{\droptitle}{-2em}

  \title{Assignment 5: Water Quality in Lakes}
    \pretitle{\vspace{\droptitle}\centering\huge}
  \posttitle{\par}
    \author{Lindsay Roth}
    \preauthor{\centering\large\emph}
  \postauthor{\par}
    \date{}
    \predate{}\postdate{}
  

\begin{document}
\maketitle

\hypertarget{overview}{%
\subsection{OVERVIEW}\label{overview}}

This exercise accompanies the lessons in Hydrologic Data Analysis on
water quality in lakes

\hypertarget{directions}{%
\subsection{Directions}\label{directions}}

\begin{enumerate}
\def\labelenumi{\arabic{enumi}.}
\tightlist
\item
  Change ``Student Name'' on line 3 (above) with your name.
\item
  Work through the steps, \textbf{creating code and output} that fulfill
  each instruction.
\item
  Be sure to \textbf{answer the questions} in this assignment document.
\item
  When you have completed the assignment, \textbf{Knit} the text and
  code into a single HTML file.
\item
  After Knitting, submit the completed exercise (HTML file) to the
  dropbox in Sakai. Add your last name into the file name (e.g.,
  ``A05\_Salk.html'') prior to submission.
\end{enumerate}

The completed exercise is due on 2 October 2019 at 9:00 am.

\hypertarget{setup}{%
\subsection{Setup}\label{setup}}

\begin{enumerate}
\def\labelenumi{\arabic{enumi}.}
\tightlist
\item
  Verify your working directory is set to the R project file,
\item
  Load the tidyverse, lubridate, and LAGOSNE packages.
\item
  Set your ggplot theme (can be theme\_classic or something else)
\item
  Load the LAGOSdata database and the trophic state index csv file we
  created on 2019/09/27.
\end{enumerate}

\begin{Shaded}
\begin{Highlighting}[]
\KeywordTok{getwd}\NormalTok{()}
\end{Highlighting}
\end{Shaded}

\begin{verbatim}
## [1] "/Users/lindsayroth/Documents/MEM 2nd Year/HydroData/Hydrologic_Data_Analysis"
\end{verbatim}

\begin{Shaded}
\begin{Highlighting}[]
\KeywordTok{library}\NormalTok{(tidyverse)}
\end{Highlighting}
\end{Shaded}

\begin{verbatim}
## -- Attaching packages ----------------------------------------------------------------------- tidyverse 1.2.1 --
\end{verbatim}

\begin{verbatim}
## v ggplot2 3.2.1     v purrr   0.3.2
## v tibble  2.1.3     v dplyr   0.8.3
## v tidyr   1.0.0     v stringr 1.4.0
## v readr   1.3.1     v forcats 0.4.0
\end{verbatim}

\begin{verbatim}
## -- Conflicts -------------------------------------------------------------------------- tidyverse_conflicts() --
## x dplyr::filter() masks stats::filter()
## x dplyr::lag()    masks stats::lag()
\end{verbatim}

\begin{Shaded}
\begin{Highlighting}[]
\KeywordTok{library}\NormalTok{(lubridate)}
\end{Highlighting}
\end{Shaded}

\begin{verbatim}
## 
## Attaching package: 'lubridate'
\end{verbatim}

\begin{verbatim}
## The following object is masked from 'package:base':
## 
##     date
\end{verbatim}

\begin{Shaded}
\begin{Highlighting}[]
\KeywordTok{library}\NormalTok{(LAGOSNE)}

\KeywordTok{theme_set}\NormalTok{(}\KeywordTok{theme_classic}\NormalTok{())}

\NormalTok{LAGOSdata <-}\StringTok{ }\KeywordTok{lagosne_load}\NormalTok{()}
\end{Highlighting}
\end{Shaded}

\begin{verbatim}
## Warning in `_f`(version = version, fpath = fpath): LAGOSNE version
## unspecified, loading version: 1.087.3
\end{verbatim}

\begin{Shaded}
\begin{Highlighting}[]
\NormalTok{TrophicData <-}\StringTok{ }\KeywordTok{read.csv}\NormalTok{(}\StringTok{"./Data/LAGOStrophic.csv"}\NormalTok{)}
\end{Highlighting}
\end{Shaded}

\hypertarget{trophic-state-index}{%
\subsection{Trophic State Index}\label{trophic-state-index}}

\begin{enumerate}
\def\labelenumi{\arabic{enumi}.}
\setcounter{enumi}{4}
\tightlist
\item
  Similar to the trophic.class column we created in class (determined
  from TSI.chl values), create two additional columns in the data frame
  that determine trophic class from TSI.secchi and TSI.tp (call these
  trophic.class.secchi and trophic.class.tp).
\end{enumerate}

\begin{Shaded}
\begin{Highlighting}[]
\NormalTok{TrophicData <-}\StringTok{ }\NormalTok{TrophicData }\OperatorTok
\StringTok{  }\KeywordTok{mutate}\NormalTok{(}\DataTypeTok{trophic.class.secchi =} 
            \KeywordTok{ifelse}\NormalTok{(TSI.secchi }\OperatorTok{<}\StringTok{ }\DecValTok{40}\NormalTok{, }\StringTok{"Oligotrophic"}\NormalTok{, }
                   \KeywordTok{ifelse}\NormalTok{(TSI.secchi }\OperatorTok{<}\StringTok{ }\DecValTok{50}\NormalTok{, }\StringTok{"Mesotrophic"}\NormalTok{,}
                          \KeywordTok{ifelse}\NormalTok{(TSI.secchi }\OperatorTok{<}\StringTok{ }\DecValTok{70}\NormalTok{, }\StringTok{"Eutrophic"}\NormalTok{, }\StringTok{"Hypereutrophic"}\NormalTok{)))) }\OperatorTok
\StringTok{  }\KeywordTok{mutate}\NormalTok{(}\DataTypeTok{trophic.class.tp =} 
            \KeywordTok{ifelse}\NormalTok{(TSI.tp }\OperatorTok{<}\StringTok{ }\DecValTok{40}\NormalTok{, }\StringTok{"Oligotrophic"}\NormalTok{, }
                   \KeywordTok{ifelse}\NormalTok{(TSI.tp }\OperatorTok{<}\StringTok{ }\DecValTok{50}\NormalTok{, }\StringTok{"Mesotrophic"}\NormalTok{,}
                          \KeywordTok{ifelse}\NormalTok{(TSI.tp }\OperatorTok{<}\StringTok{ }\DecValTok{70}\NormalTok{, }\StringTok{"Eutrophic"}\NormalTok{, }\StringTok{"Hypereutrophic"}\NormalTok{))))}
\end{Highlighting}
\end{Shaded}

\begin{enumerate}
\def\labelenumi{\arabic{enumi}.}
\setcounter{enumi}{5}
\tightlist
\item
  How many observations fall into the four trophic state categories for
  the three metrics (trophic.class, trophic.class.secchi,
  trophic.class.tp)? Hint: \texttt{count} function.
\end{enumerate}

\begin{Shaded}
\begin{Highlighting}[]
\KeywordTok{count}\NormalTok{(TrophicData, trophic.class)}
\end{Highlighting}
\end{Shaded}

\begin{verbatim}
## # A tibble: 4 x 2
##   trophic.class      n
##   <fct>          <int>
## 1 Eutrophic      41861
## 2 Hypereutrophic 14379
## 3 Mesotrophic    15413
## 4 Oligotrophic    3298
\end{verbatim}

\begin{Shaded}
\begin{Highlighting}[]
\KeywordTok{count}\NormalTok{(TrophicData, trophic.class.secchi)}
\end{Highlighting}
\end{Shaded}

\begin{verbatim}
## # A tibble: 4 x 2
##   trophic.class.secchi     n
##   <chr>                <int>
## 1 Eutrophic            28659
## 2 Hypereutrophic        5099
## 3 Mesotrophic          25083
## 4 Oligotrophic         16110
\end{verbatim}

\begin{Shaded}
\begin{Highlighting}[]
\KeywordTok{count}\NormalTok{(TrophicData, trophic.class.tp)}
\end{Highlighting}
\end{Shaded}

\begin{verbatim}
## # A tibble: 4 x 2
##   trophic.class.tp     n
##   <chr>            <int>
## 1 Eutrophic        24839
## 2 Hypereutrophic    7228
## 3 Mesotrophic      23023
## 4 Oligotrophic     19861
\end{verbatim}

\begin{enumerate}
\def\labelenumi{\arabic{enumi}.}
\setcounter{enumi}{6}
\tightlist
\item
  What proportion of total observations are considered eutrohic or
  hypereutrophic according to the three different metrics
  (trophic.class, trophic.class.secchi, trophic.class.tp)?
\end{enumerate}

\begin{Shaded}
\begin{Highlighting}[]
\CommentTok{#trophic.class}
\CommentTok{##Hypereutrophic}
\DecValTok{14379}\OperatorTok{/}\DecValTok{74951}
\end{Highlighting}
\end{Shaded}

\begin{verbatim}
## [1] 0.1918453
\end{verbatim}

\begin{Shaded}
\begin{Highlighting}[]
\CommentTok{##0.19 or 19%}

\CommentTok{##Eutrophic}
\DecValTok{41861}\OperatorTok{/}\DecValTok{74951}
\end{Highlighting}
\end{Shaded}

\begin{verbatim}
## [1] 0.5585116
\end{verbatim}

\begin{Shaded}
\begin{Highlighting}[]
\CommentTok{##0.56 or 56%}


\CommentTok{#trophic.class.secchi}
\CommentTok{##Hypereutrophic}
\DecValTok{5099}\OperatorTok{/}\DecValTok{74951}
\end{Highlighting}
\end{Shaded}

\begin{verbatim}
## [1] 0.06803111
\end{verbatim}

\begin{Shaded}
\begin{Highlighting}[]
\CommentTok{##0.07 or 7%}

\CommentTok{##Eutrophic}
\DecValTok{28659}\OperatorTok{/}\DecValTok{74951}
\end{Highlighting}
\end{Shaded}

\begin{verbatim}
## [1] 0.3823698
\end{verbatim}

\begin{Shaded}
\begin{Highlighting}[]
\CommentTok{##0.38 or 38%}


\CommentTok{#trophic.class.tp}
\CommentTok{##Hypereutrophic}
\DecValTok{7228}\OperatorTok{/}\DecValTok{74951}
\end{Highlighting}
\end{Shaded}

\begin{verbatim}
## [1] 0.09643634
\end{verbatim}

\begin{Shaded}
\begin{Highlighting}[]
\CommentTok{##0.096 or 9.6%}

\CommentTok{##Eutrophic}
\DecValTok{24839}\OperatorTok{/}\DecValTok{74951}
\end{Highlighting}
\end{Shaded}

\begin{verbatim}
## [1] 0.3314032
\end{verbatim}

\begin{Shaded}
\begin{Highlighting}[]
\CommentTok{##0.33 or 33%}
\end{Highlighting}
\end{Shaded}

Which of these metrics is most conservative in its designation of
eutrophic conditions? Why might this be?

\begin{quote}
Total phosphorus seems to be the most conservativie in designating
eutrophic conditions. This may be because there are other factors
influencing higher levels of chlorophyll a and secchi depth, such as
total nitrogen, turbidity, sediment load, etc. Total phosphorus levels
are not influenced by any of these factors.
\end{quote}

Note: To take this further, a researcher might determine which trophic
classes are susceptible to being differently categorized by the
different metrics and whether certain metrics are prone to categorizing
trophic class as more or less eutrophic. This would entail more complex
code.

\hypertarget{nutrient-concentrations}{%
\subsection{Nutrient Concentrations}\label{nutrient-concentrations}}

\begin{enumerate}
\def\labelenumi{\arabic{enumi}.}
\setcounter{enumi}{7}
\tightlist
\item
  Create a data frame that includes the columns lagoslakeid, sampledate,
  tn, tp, state, and state\_name. Mutate this data frame to include
  sampleyear and samplemonth columns as well. Call this data frame
  LAGOSNandP.
\end{enumerate}

\begin{Shaded}
\begin{Highlighting}[]
\NormalTok{LAGOSlocus <-}\StringTok{ }\NormalTok{LAGOSdata}\OperatorTok{$}\NormalTok{locus}
\NormalTok{LAGOSstate <-}\StringTok{ }\NormalTok{LAGOSdata}\OperatorTok{$}\NormalTok{state}
\NormalTok{LAGOSnutrient <-}\StringTok{ }\NormalTok{LAGOSdata}\OperatorTok{$}\NormalTok{epi_nutr}

\NormalTok{LAGOSlocus}\OperatorTok{$}\NormalTok{lagoslakeid <-}\StringTok{ }\KeywordTok{as.factor}\NormalTok{(LAGOSlocus}\OperatorTok{$}\NormalTok{lagoslakeid)}
\NormalTok{LAGOSnutrient}\OperatorTok{$}\NormalTok{lagoslakeid <-}\StringTok{ }\KeywordTok{as.factor}\NormalTok{(LAGOSnutrient}\OperatorTok{$}\NormalTok{lagoslakeid)}

\NormalTok{LAGOSlocations <-}\StringTok{ }\KeywordTok{left_join}\NormalTok{(LAGOSlocus, LAGOSstate, }\DataTypeTok{by =} \StringTok{"state_zoneid"}\NormalTok{)}

\NormalTok{LAGOSlocations <-}\StringTok{ }
\StringTok{  }\KeywordTok{within}\NormalTok{(LAGOSlocations, }
\NormalTok{         state <-}\StringTok{ }\KeywordTok{factor}\NormalTok{(state, }\DataTypeTok{levels =} \KeywordTok{names}\NormalTok{(}\KeywordTok{sort}\NormalTok{(}\KeywordTok{table}\NormalTok{(state), }\DataTypeTok{decreasing=}\OtherTok{TRUE}\NormalTok{))))}

\NormalTok{LAGOSNandP <-}\StringTok{ }
\StringTok{  }\KeywordTok{left_join}\NormalTok{(LAGOSnutrient, LAGOSlocations, }\DataTypeTok{by =} \StringTok{"lagoslakeid"}\NormalTok{) }\OperatorTok
\StringTok{  }\KeywordTok{select}\NormalTok{( lagoslakeid, sampledate, tn, tp, state, state_name) }\OperatorTok
\StringTok{  }\KeywordTok{mutate}\NormalTok{(}\DataTypeTok{sampleyear =} \KeywordTok{year}\NormalTok{(sampledate), }
         \DataTypeTok{samplemonth =} \KeywordTok{month}\NormalTok{(sampledate))}
\end{Highlighting}
\end{Shaded}

\begin{verbatim}
## Warning: Column `lagoslakeid` joining factors with different levels,
## coercing to character vector
\end{verbatim}

\begin{enumerate}
\def\labelenumi{\arabic{enumi}.}
\setcounter{enumi}{8}
\tightlist
\item
  Create two violin plots comparing TN and TP concentrations across
  states. Include a 50th percentile line inside the violins.
\end{enumerate}

\begin{Shaded}
\begin{Highlighting}[]
\NormalTok{NPlot <-}\StringTok{ }\KeywordTok{ggplot}\NormalTok{(LAGOSNandP) }\OperatorTok{+}
\StringTok{  }\KeywordTok{geom_violin}\NormalTok{(}\KeywordTok{aes}\NormalTok{(}\DataTypeTok{x =}\NormalTok{ state_name, }\DataTypeTok{y =}\NormalTok{ tn), }\DataTypeTok{draw_quantiles =} \FloatTok{0.5}\NormalTok{) }\OperatorTok{+}
\StringTok{  }\KeywordTok{labs}\NormalTok{(}\DataTypeTok{x =} \StringTok{"State"}\NormalTok{, }\DataTypeTok{y =}  \KeywordTok{expression}\NormalTok{(}\StringTok{"Total Nitrogen"}\NormalTok{ (mu}\OperatorTok{*}\NormalTok{g }\OperatorTok{/}\StringTok{ }\NormalTok{L))) }\OperatorTok{+}
\StringTok{  }\KeywordTok{theme}\NormalTok{(}\DataTypeTok{axis.text.x =} \KeywordTok{element_text}\NormalTok{(}\DataTypeTok{angle =} \DecValTok{45}\NormalTok{, }\DataTypeTok{hjust =} \DecValTok{1}\NormalTok{))}
\KeywordTok{print}\NormalTok{(NPlot)}
\end{Highlighting}
\end{Shaded}

\begin{verbatim}
## Warning: Removed 774226 rows containing non-finite values (stat_ydensity).
\end{verbatim}

\begin{verbatim}
## Warning in regularize.values(x, y, ties, missing(ties)): collapsing to
## unique 'x' values

## Warning in regularize.values(x, y, ties, missing(ties)): collapsing to
## unique 'x' values

## Warning in regularize.values(x, y, ties, missing(ties)): collapsing to
## unique 'x' values

## Warning in regularize.values(x, y, ties, missing(ties)): collapsing to
## unique 'x' values

## Warning in regularize.values(x, y, ties, missing(ties)): collapsing to
## unique 'x' values

## Warning in regularize.values(x, y, ties, missing(ties)): collapsing to
## unique 'x' values
\end{verbatim}

\includegraphics{A05_Roth_files/figure-latex/unnamed-chunk-5-1.pdf}

\begin{Shaded}
\begin{Highlighting}[]
\NormalTok{PPlot <-}\StringTok{ }\KeywordTok{ggplot}\NormalTok{(LAGOSNandP) }\OperatorTok{+}
\StringTok{  }\KeywordTok{geom_violin}\NormalTok{(}\KeywordTok{aes}\NormalTok{(}\DataTypeTok{x =}\NormalTok{ state_name, }\DataTypeTok{y =}\NormalTok{ tp), }\DataTypeTok{draw_quantiles =} \FloatTok{0.5}\NormalTok{) }\OperatorTok{+}
\StringTok{  }\KeywordTok{labs}\NormalTok{(}\DataTypeTok{x =} \StringTok{"State"}\NormalTok{, }\DataTypeTok{y =}  \KeywordTok{expression}\NormalTok{(}\StringTok{"Total Phosphorus"}\NormalTok{ (mu}\OperatorTok{*}\NormalTok{g }\OperatorTok{/}\StringTok{ }\NormalTok{L))) }\OperatorTok{+}
\StringTok{  }\KeywordTok{theme}\NormalTok{(}\DataTypeTok{axis.text.x =} \KeywordTok{element_text}\NormalTok{(}\DataTypeTok{angle =} \DecValTok{45}\NormalTok{, }\DataTypeTok{hjust =} \DecValTok{1}\NormalTok{))}
\KeywordTok{print}\NormalTok{(PPlot)}
\end{Highlighting}
\end{Shaded}

\begin{verbatim}
## Warning: Removed 672861 rows containing non-finite values (stat_ydensity).

## Warning: collapsing to unique 'x' values

## Warning: collapsing to unique 'x' values

## Warning: collapsing to unique 'x' values

## Warning: collapsing to unique 'x' values

## Warning: collapsing to unique 'x' values

## Warning: collapsing to unique 'x' values

## Warning: collapsing to unique 'x' values

## Warning: collapsing to unique 'x' values

## Warning: collapsing to unique 'x' values

## Warning: collapsing to unique 'x' values

## Warning: collapsing to unique 'x' values

## Warning: collapsing to unique 'x' values

## Warning: collapsing to unique 'x' values
\end{verbatim}

\includegraphics{A05_Roth_files/figure-latex/unnamed-chunk-5-2.pdf}

Which states have the highest and lowest median concentrations?

\begin{quote}
TN: Iowa and Ohio have the highest, New Hampshire Maine and Vermont have
the lowest
\end{quote}

\begin{quote}
TP: Iowa and Illinois have the highest, Maine Michigan and New Hampshire
have the lowest
\end{quote}

Which states have the highest and lowest concentration ranges?

\begin{quote}
TN: Iowa has the highest range, New Hampshire has the lowest range
\end{quote}

\begin{quote}
TP: Illinois has the highest range, Pennsylvania has the lowest range
\end{quote}

\begin{enumerate}
\def\labelenumi{\arabic{enumi}.}
\setcounter{enumi}{9}
\tightlist
\item
  Create two jitter plots comparing TN and TP concentrations across
  states, with samplemonth as the color. Choose a color palette other
  than the ggplot default.
\end{enumerate}

\begin{Shaded}
\begin{Highlighting}[]
\NormalTok{N.by.Month <-}\StringTok{  }
\KeywordTok{ggplot}\NormalTok{(LAGOSNandP, }
       \KeywordTok{aes}\NormalTok{(}\DataTypeTok{x =} \KeywordTok{as.factor}\NormalTok{(state_name), }\DataTypeTok{y =}\NormalTok{ tn, }\DataTypeTok{color =}\NormalTok{ samplemonth)) }\OperatorTok{+}
\StringTok{  }\KeywordTok{geom_jitter}\NormalTok{(}\DataTypeTok{alpha =} \FloatTok{0.2}\NormalTok{) }\OperatorTok{+}\StringTok{ }
\StringTok{  }\KeywordTok{labs}\NormalTok{(}\DataTypeTok{x =} \StringTok{"State"}\NormalTok{, }\DataTypeTok{y =} \KeywordTok{expression}\NormalTok{(}\StringTok{"Total Nitrogen"}\NormalTok{(mu}\OperatorTok{*}\NormalTok{g }\OperatorTok{/}\StringTok{ }\NormalTok{L)), }\DataTypeTok{color =} \StringTok{"Month"}\NormalTok{) }\OperatorTok{+}
\StringTok{  }\KeywordTok{scale_color_viridis_c}\NormalTok{(}\DataTypeTok{option =} \StringTok{"magma"}\NormalTok{) }\OperatorTok{+}
\StringTok{  }\KeywordTok{theme}\NormalTok{(}\DataTypeTok{axis.text.x =} \KeywordTok{element_text}\NormalTok{(}\DataTypeTok{angle =} \DecValTok{45}\NormalTok{, }\DataTypeTok{hjust =} \DecValTok{1}\NormalTok{))}
\KeywordTok{print}\NormalTok{(N.by.Month)}
\end{Highlighting}
\end{Shaded}

\begin{verbatim}
## Warning: Removed 774226 rows containing missing values (geom_point).
\end{verbatim}

\includegraphics{A05_Roth_files/figure-latex/unnamed-chunk-6-1.pdf}

\begin{Shaded}
\begin{Highlighting}[]
\NormalTok{P.by.Month <-}\StringTok{  }
\KeywordTok{ggplot}\NormalTok{(LAGOSNandP, }
       \KeywordTok{aes}\NormalTok{(}\DataTypeTok{x =} \KeywordTok{as.factor}\NormalTok{(state_name), }\DataTypeTok{y =}\NormalTok{ tp, }\DataTypeTok{color =}\NormalTok{ samplemonth)) }\OperatorTok{+}
\StringTok{  }\KeywordTok{geom_jitter}\NormalTok{(}\DataTypeTok{alpha =} \FloatTok{0.2}\NormalTok{) }\OperatorTok{+}\StringTok{ }
\StringTok{  }\KeywordTok{labs}\NormalTok{(}\DataTypeTok{x =} \StringTok{"State"}\NormalTok{, }\DataTypeTok{y =} \KeywordTok{expression}\NormalTok{(}\StringTok{"Total Phosphorus"}\NormalTok{(mu}\OperatorTok{*}\NormalTok{g }\OperatorTok{/}\StringTok{ }\NormalTok{L)), }\DataTypeTok{color =} \StringTok{"Month"}\NormalTok{) }\OperatorTok{+}
\StringTok{  }\KeywordTok{scale_color_viridis_c}\NormalTok{(}\DataTypeTok{option =} \StringTok{"magma"}\NormalTok{) }\OperatorTok{+}
\StringTok{  }\KeywordTok{theme}\NormalTok{(}\DataTypeTok{axis.text.x =} \KeywordTok{element_text}\NormalTok{(}\DataTypeTok{angle =} \DecValTok{45}\NormalTok{, }\DataTypeTok{hjust =} \DecValTok{1}\NormalTok{))}
\KeywordTok{print}\NormalTok{(P.by.Month)}
\end{Highlighting}
\end{Shaded}

\begin{verbatim}
## Warning: Removed 672861 rows containing missing values (geom_point).
\end{verbatim}

\includegraphics{A05_Roth_files/figure-latex/unnamed-chunk-6-2.pdf}

Which states have the most samples? How might this have impacted total
ranges from \#9?

\begin{quote}
TN: It looks like Iowa, Ohio, Minnesota, and Missouri have taken the
most tn samples. This may have contributed to the wide range of values
for Iowa and Ohio.
\end{quote}

\begin{quote}
TP: Illinois, Iowa, Minnesota, Missouri, and Wisconsive have taken the
most tp samples. This may have contributed to the wide range of values
for Illinois, Minnesota, and Wisconsin.
\end{quote}

Which months are sampled most extensively? Does this differ among
states?

\begin{quote}
TN: It looks like May, June, July, and August were most extensively
sampled. Some states like Indiana appear to have only sampled during one
month (likely July), while other states have more samples later in the
year, like New York and Rhode Island as well as earlier in the year like
Michigan and Ohio. Not many samples in any states were taken in the
winter months.
\end{quote}

\begin{quote}
TP: Phosphorus is similar to Nitrogen. THe most extensively sampled
months were June, July, and August, with Indiana appearing to only
sample in July and August while Michigan, Vermont, and Wisconsin have
samples from earlier and later in the year. Not many samples in any
states were taken in the winter months.
\end{quote}

\begin{enumerate}
\def\labelenumi{\arabic{enumi}.}
\setcounter{enumi}{10}
\tightlist
\item
  Create two jitter plots comparing TN and TP concentrations across
  states, with sampleyear as the color. Choose a color palette other
  than the ggplot default.
\end{enumerate}

\begin{Shaded}
\begin{Highlighting}[]
\NormalTok{N.by.Year <-}\StringTok{  }
\KeywordTok{ggplot}\NormalTok{(LAGOSNandP, }
       \KeywordTok{aes}\NormalTok{(}\DataTypeTok{x =} \KeywordTok{as.factor}\NormalTok{(state_name), }\DataTypeTok{y =}\NormalTok{ tn, }\DataTypeTok{color =}\NormalTok{ sampleyear)) }\OperatorTok{+}
\StringTok{  }\KeywordTok{geom_jitter}\NormalTok{(}\DataTypeTok{alpha =} \FloatTok{0.2}\NormalTok{) }\OperatorTok{+}\StringTok{ }
\StringTok{  }\KeywordTok{labs}\NormalTok{(}\DataTypeTok{x =} \StringTok{"State"}\NormalTok{, }\DataTypeTok{y =} \KeywordTok{expression}\NormalTok{(}\StringTok{"Total Nitrogen"}\NormalTok{(mu}\OperatorTok{*}\NormalTok{g }\OperatorTok{/}\StringTok{ }\NormalTok{L)), }\DataTypeTok{color =} \StringTok{"Year"}\NormalTok{) }\OperatorTok{+}
\StringTok{  }\KeywordTok{scale_color_viridis_c}\NormalTok{(}\DataTypeTok{option =} \StringTok{"magma"}\NormalTok{) }\OperatorTok{+}
\StringTok{  }\KeywordTok{theme}\NormalTok{(}\DataTypeTok{axis.text.x =} \KeywordTok{element_text}\NormalTok{(}\DataTypeTok{angle =} \DecValTok{45}\NormalTok{, }\DataTypeTok{hjust =} \DecValTok{1}\NormalTok{))}
\KeywordTok{print}\NormalTok{(N.by.Year)}
\end{Highlighting}
\end{Shaded}

\begin{verbatim}
## Warning: Removed 774226 rows containing missing values (geom_point).
\end{verbatim}

\includegraphics{A05_Roth_files/figure-latex/unnamed-chunk-7-1.pdf}

\begin{Shaded}
\begin{Highlighting}[]
\NormalTok{P.by.Year <-}\StringTok{  }
\KeywordTok{ggplot}\NormalTok{(LAGOSNandP, }
       \KeywordTok{aes}\NormalTok{(}\DataTypeTok{x =} \KeywordTok{as.factor}\NormalTok{(state_name), }\DataTypeTok{y =}\NormalTok{ tp, }\DataTypeTok{color =}\NormalTok{ sampleyear)) }\OperatorTok{+}
\StringTok{  }\KeywordTok{geom_jitter}\NormalTok{(}\DataTypeTok{alpha =} \FloatTok{0.2}\NormalTok{) }\OperatorTok{+}\StringTok{ }
\StringTok{  }\KeywordTok{labs}\NormalTok{(}\DataTypeTok{x =} \StringTok{"State"}\NormalTok{, }\DataTypeTok{y =} \KeywordTok{expression}\NormalTok{(}\StringTok{"Total Phosphorus"}\NormalTok{(mu}\OperatorTok{*}\NormalTok{g }\OperatorTok{/}\StringTok{ }\NormalTok{L)), }\DataTypeTok{color =} \StringTok{"Year"}\NormalTok{) }\OperatorTok{+}
\StringTok{  }\KeywordTok{scale_color_viridis_c}\NormalTok{(}\DataTypeTok{option =} \StringTok{"magma"}\NormalTok{) }\OperatorTok{+}
\StringTok{  }\KeywordTok{theme}\NormalTok{(}\DataTypeTok{axis.text.x =} \KeywordTok{element_text}\NormalTok{(}\DataTypeTok{angle =} \DecValTok{45}\NormalTok{, }\DataTypeTok{hjust =} \DecValTok{1}\NormalTok{))}
\KeywordTok{print}\NormalTok{(P.by.Year)}
\end{Highlighting}
\end{Shaded}

\begin{verbatim}
## Warning: Removed 672861 rows containing missing values (geom_point).
\end{verbatim}

\includegraphics{A05_Roth_files/figure-latex/unnamed-chunk-7-2.pdf}

Which years are sampled most extensively? Does this differ among states?

\begin{quote}
TN: The years that were sampled most extensively include the late 1990s
and 2000s. Connecticut has some mesurements from what appears to be the
1970s and 1980s.
\end{quote}

\begin{quote}
TP: The Years that were sampled most were the 1990s and the 2000s.
Minnesota also has samples starting from what looks like the 1950s.
Connecticut, Maine, New York, Vermont, and Wisconsin have samples from
the 1980s.
\end{quote}

\hypertarget{reflection}{%
\subsection{Reflection}\label{reflection}}

\begin{enumerate}
\def\labelenumi{\arabic{enumi}.}
\setcounter{enumi}{11}
\tightlist
\item
  What are 2-3 conclusions or summary points about lake water quality
  you learned through your analysis?
\end{enumerate}

\begin{quote}
Conclusion 1: States with large agricultural economies have higher
measurements of total nitrogen and total phosphorus than states with
smaller agricultural economies. Coclusion 2: Total phosphorus, secchi
depth, and chlorophyll a all predict slightly varied eutrophic levels in
water bodies.
\end{quote}

\begin{enumerate}
\def\labelenumi{\arabic{enumi}.}
\setcounter{enumi}{12}
\tightlist
\item
  What data, visualizations, and/or models supported your conclusions
  from 12?
\end{enumerate}

\begin{quote}
In the jitter plots, the highest measurements of total nitrogen and
total phosphorus were seen in Illinois, Indiana, Minnesota, Missouri,
Ohio and Wisconsin.
\end{quote}

\begin{enumerate}
\def\labelenumi{\arabic{enumi}.}
\setcounter{enumi}{13}
\tightlist
\item
  Did hands-on data analysis impact your learning about water quality
  relative to a theory-based lesson? If so, how?
\end{enumerate}

\begin{quote}
Yes, being able to visualize trends helps me learn concepts much more
than just reading explanations.
\end{quote}

\begin{enumerate}
\def\labelenumi{\arabic{enumi}.}
\setcounter{enumi}{14}
\tightlist
\item
  How did the real-world data compare with your expectations from
  theory?
\end{enumerate}

\begin{quote}
The data we used for this assignment aligned fairly well with my
expectations. I know that a lot of nitrogen and phosphorus runs off from
agricultural land due to excess fertilizers applied on the surface of
the soil. These nutrients, when added to lakes, rivers, and
estuaries,contribute to higher biological activity in these water
bodies. Therefore, I expected states with higher agricultural activity
to have higher trophic levels.
\end{quote}


\end{document}
