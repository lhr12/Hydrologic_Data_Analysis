\documentclass[]{article}
\usepackage{lmodern}
\usepackage{amssymb,amsmath}
\usepackage{ifxetex,ifluatex}
\usepackage{fixltx2e} % provides \textsubscript
\ifnum 0\ifxetex 1\fi\ifluatex 1\fi=0 % if pdftex
  \usepackage[T1]{fontenc}
  \usepackage[utf8]{inputenc}
\else % if luatex or xelatex
  \ifxetex
    \usepackage{mathspec}
  \else
    \usepackage{fontspec}
  \fi
  \defaultfontfeatures{Ligatures=TeX,Scale=MatchLowercase}
\fi
% use upquote if available, for straight quotes in verbatim environments
\IfFileExists{upquote.sty}{\usepackage{upquote}}{}
% use microtype if available
\IfFileExists{microtype.sty}{%
\usepackage{microtype}
\UseMicrotypeSet[protrusion]{basicmath} % disable protrusion for tt fonts
}{}
\usepackage[margin=2.54cm]{geometry}
\usepackage{hyperref}
\hypersetup{unicode=true,
            pdftitle={Assignment 6: Time Series Analysis},
            pdfauthor={Lindsay Roth},
            pdfborder={0 0 0},
            breaklinks=true}
\urlstyle{same}  % don't use monospace font for urls
\usepackage{color}
\usepackage{fancyvrb}
\newcommand{\VerbBar}{|}
\newcommand{\VERB}{\Verb[commandchars=\\\{\}]}
\DefineVerbatimEnvironment{Highlighting}{Verbatim}{commandchars=\\\{\}}
% Add ',fontsize=\small' for more characters per line
\usepackage{framed}
\definecolor{shadecolor}{RGB}{248,248,248}
\newenvironment{Shaded}{\begin{snugshade}}{\end{snugshade}}
\newcommand{\AlertTok}[1]{\textcolor[rgb]{0.94,0.16,0.16}{#1}}
\newcommand{\AnnotationTok}[1]{\textcolor[rgb]{0.56,0.35,0.01}{\textbf{\textit{#1}}}}
\newcommand{\AttributeTok}[1]{\textcolor[rgb]{0.77,0.63,0.00}{#1}}
\newcommand{\BaseNTok}[1]{\textcolor[rgb]{0.00,0.00,0.81}{#1}}
\newcommand{\BuiltInTok}[1]{#1}
\newcommand{\CharTok}[1]{\textcolor[rgb]{0.31,0.60,0.02}{#1}}
\newcommand{\CommentTok}[1]{\textcolor[rgb]{0.56,0.35,0.01}{\textit{#1}}}
\newcommand{\CommentVarTok}[1]{\textcolor[rgb]{0.56,0.35,0.01}{\textbf{\textit{#1}}}}
\newcommand{\ConstantTok}[1]{\textcolor[rgb]{0.00,0.00,0.00}{#1}}
\newcommand{\ControlFlowTok}[1]{\textcolor[rgb]{0.13,0.29,0.53}{\textbf{#1}}}
\newcommand{\DataTypeTok}[1]{\textcolor[rgb]{0.13,0.29,0.53}{#1}}
\newcommand{\DecValTok}[1]{\textcolor[rgb]{0.00,0.00,0.81}{#1}}
\newcommand{\DocumentationTok}[1]{\textcolor[rgb]{0.56,0.35,0.01}{\textbf{\textit{#1}}}}
\newcommand{\ErrorTok}[1]{\textcolor[rgb]{0.64,0.00,0.00}{\textbf{#1}}}
\newcommand{\ExtensionTok}[1]{#1}
\newcommand{\FloatTok}[1]{\textcolor[rgb]{0.00,0.00,0.81}{#1}}
\newcommand{\FunctionTok}[1]{\textcolor[rgb]{0.00,0.00,0.00}{#1}}
\newcommand{\ImportTok}[1]{#1}
\newcommand{\InformationTok}[1]{\textcolor[rgb]{0.56,0.35,0.01}{\textbf{\textit{#1}}}}
\newcommand{\KeywordTok}[1]{\textcolor[rgb]{0.13,0.29,0.53}{\textbf{#1}}}
\newcommand{\NormalTok}[1]{#1}
\newcommand{\OperatorTok}[1]{\textcolor[rgb]{0.81,0.36,0.00}{\textbf{#1}}}
\newcommand{\OtherTok}[1]{\textcolor[rgb]{0.56,0.35,0.01}{#1}}
\newcommand{\PreprocessorTok}[1]{\textcolor[rgb]{0.56,0.35,0.01}{\textit{#1}}}
\newcommand{\RegionMarkerTok}[1]{#1}
\newcommand{\SpecialCharTok}[1]{\textcolor[rgb]{0.00,0.00,0.00}{#1}}
\newcommand{\SpecialStringTok}[1]{\textcolor[rgb]{0.31,0.60,0.02}{#1}}
\newcommand{\StringTok}[1]{\textcolor[rgb]{0.31,0.60,0.02}{#1}}
\newcommand{\VariableTok}[1]{\textcolor[rgb]{0.00,0.00,0.00}{#1}}
\newcommand{\VerbatimStringTok}[1]{\textcolor[rgb]{0.31,0.60,0.02}{#1}}
\newcommand{\WarningTok}[1]{\textcolor[rgb]{0.56,0.35,0.01}{\textbf{\textit{#1}}}}
\usepackage{graphicx,grffile}
\makeatletter
\def\maxwidth{\ifdim\Gin@nat@width>\linewidth\linewidth\else\Gin@nat@width\fi}
\def\maxheight{\ifdim\Gin@nat@height>\textheight\textheight\else\Gin@nat@height\fi}
\makeatother
% Scale images if necessary, so that they will not overflow the page
% margins by default, and it is still possible to overwrite the defaults
% using explicit options in \includegraphics[width, height, ...]{}
\setkeys{Gin}{width=\maxwidth,height=\maxheight,keepaspectratio}
\IfFileExists{parskip.sty}{%
\usepackage{parskip}
}{% else
\setlength{\parindent}{0pt}
\setlength{\parskip}{6pt plus 2pt minus 1pt}
}
\setlength{\emergencystretch}{3em}  % prevent overfull lines
\providecommand{\tightlist}{%
  \setlength{\itemsep}{0pt}\setlength{\parskip}{0pt}}
\setcounter{secnumdepth}{0}
% Redefines (sub)paragraphs to behave more like sections
\ifx\paragraph\undefined\else
\let\oldparagraph\paragraph
\renewcommand{\paragraph}[1]{\oldparagraph{#1}\mbox{}}
\fi
\ifx\subparagraph\undefined\else
\let\oldsubparagraph\subparagraph
\renewcommand{\subparagraph}[1]{\oldsubparagraph{#1}\mbox{}}
\fi

%%% Use protect on footnotes to avoid problems with footnotes in titles
\let\rmarkdownfootnote\footnote%
\def\footnote{\protect\rmarkdownfootnote}

%%% Change title format to be more compact
\usepackage{titling}

% Create subtitle command for use in maketitle
\providecommand{\subtitle}[1]{
  \posttitle{
    \begin{center}\large#1\end{center}
    }
}

\setlength{\droptitle}{-2em}

  \title{Assignment 6: Time Series Analysis}
    \pretitle{\vspace{\droptitle}\centering\huge}
  \posttitle{\par}
    \author{Lindsay Roth}
    \preauthor{\centering\large\emph}
  \postauthor{\par}
    \date{}
    \predate{}\postdate{}
  

\begin{document}
\maketitle

\hypertarget{overview}{%
\subsection{OVERVIEW}\label{overview}}

This exercise accompanies the lessons in Hydrologic Data Analysis on
time series analysis

\hypertarget{directions}{%
\subsection{Directions}\label{directions}}

\begin{enumerate}
\def\labelenumi{\arabic{enumi}.}
\tightlist
\item
  Change ``Student Name'' on line 3 (above) with your name.
\item
  Work through the steps, \textbf{creating code and output} that fulfill
  each instruction.
\item
  Be sure to \textbf{answer the questions} in this assignment document.
\item
  When you have completed the assignment, \textbf{Knit} the text and
  code into a single pdf file.
\item
  After Knitting, submit the completed exercise (pdf file) to the
  dropbox in Sakai. Add your last name into the file name (e.g.,
  ``A06\_Salk.html'') prior to submission.
\end{enumerate}

The completed exercise is due on 11 October 2019 at 9:00 am.

\hypertarget{setup}{%
\subsection{Setup}\label{setup}}

\begin{enumerate}
\def\labelenumi{\arabic{enumi}.}
\tightlist
\item
  Verify your working directory is set to the R project file,
\item
  Load the tidyverse, lubridate, trend, and dataRetrieval packages.
\item
  Set your ggplot theme (can be theme\_classic or something else)
\item
  Load the ClearCreekDischarge.Monthly.csv file from the processed data
  folder. Call this data frame ClearCreekDischarge.Monthly.
\end{enumerate}

\begin{Shaded}
\begin{Highlighting}[]
\KeywordTok{getwd}\NormalTok{()}
\end{Highlighting}
\end{Shaded}

\begin{verbatim}
## [1] "/Users/lindsayroth/Documents/MEM 2nd Year/HydroData/Hydrologic_Data_Analysis"
\end{verbatim}

\begin{Shaded}
\begin{Highlighting}[]
\KeywordTok{library}\NormalTok{(tidyverse)}
\end{Highlighting}
\end{Shaded}

\begin{verbatim}
## -- Attaching packages ------------------------------------------------------------------------ tidyverse 1.2.1 --
\end{verbatim}

\begin{verbatim}
## v ggplot2 3.2.1     v purrr   0.3.2
## v tibble  2.1.3     v dplyr   0.8.3
## v tidyr   1.0.0     v stringr 1.4.0
## v readr   1.3.1     v forcats 0.4.0
\end{verbatim}

\begin{verbatim}
## -- Conflicts --------------------------------------------------------------------------- tidyverse_conflicts() --
## x dplyr::filter() masks stats::filter()
## x dplyr::lag()    masks stats::lag()
\end{verbatim}

\begin{Shaded}
\begin{Highlighting}[]
\KeywordTok{library}\NormalTok{(lubridate)}
\end{Highlighting}
\end{Shaded}

\begin{verbatim}
## 
## Attaching package: 'lubridate'
\end{verbatim}

\begin{verbatim}
## The following object is masked from 'package:base':
## 
##     date
\end{verbatim}

\begin{Shaded}
\begin{Highlighting}[]
\KeywordTok{library}\NormalTok{(trend)}
\KeywordTok{library}\NormalTok{(dataRetrieval)}

\KeywordTok{theme_set}\NormalTok{(}\KeywordTok{theme_classic}\NormalTok{())}

\NormalTok{ClearCreekDischarge.Monthly <-}\StringTok{ }\KeywordTok{read.csv}\NormalTok{(}\StringTok{"./Data/Processed/ClearCreekDischarge.Monthly.csv"}\NormalTok{)}
\end{Highlighting}
\end{Shaded}

\hypertarget{time-series-decomposition}{%
\subsection{Time Series Decomposition}\label{time-series-decomposition}}

\begin{enumerate}
\def\labelenumi{\arabic{enumi}.}
\setcounter{enumi}{4}
\tightlist
\item
  Create a new data frame that includes daily mean discharge at the Eno
  River for all available dates (\texttt{siteNumbers\ =\ "02085070"}).
  Rename the columns accordingly.
\item
  Plot discharge over time with geom\_line. Make sure axis labels are
  formatted appropriately.
\item
  Create a time series of discharge
\item
  Decompose the time series using the \texttt{stl} function.
\item
  Visualize the decomposed time series.
\end{enumerate}

\begin{Shaded}
\begin{Highlighting}[]
\NormalTok{EnoDischarge <-}\StringTok{ }\KeywordTok{readNWISdv}\NormalTok{(}\DataTypeTok{siteNumbers =} \StringTok{"02085070"}\NormalTok{,}
                     \DataTypeTok{parameterCd =} \StringTok{"00060"}\NormalTok{, }\CommentTok{# discharge (ft3/s)}
                     \DataTypeTok{startDate =} \StringTok{""}\NormalTok{,}
                     \DataTypeTok{endDate =} \StringTok{""}\NormalTok{)}

\KeywordTok{names}\NormalTok{(EnoDischarge)[}\DecValTok{4}\OperatorTok{:}\DecValTok{5}\NormalTok{] <-}\StringTok{ }\KeywordTok{c}\NormalTok{(}\StringTok{"Discharge"}\NormalTok{, }\StringTok{"Approval.Code"}\NormalTok{)}

\NormalTok{EnoDischargePlot <-}\StringTok{ }\KeywordTok{ggplot}\NormalTok{(EnoDischarge) }\OperatorTok{+}
\StringTok{  }\KeywordTok{geom_line}\NormalTok{(}\KeywordTok{aes}\NormalTok{(}\DataTypeTok{x =}\NormalTok{ Date, }\DataTypeTok{y =}\NormalTok{ Discharge)) }\OperatorTok{+}
\StringTok{  }\KeywordTok{labs}\NormalTok{(}\DataTypeTok{x =} \StringTok{"Measurement Dates 1963-2019"}\NormalTok{, }\DataTypeTok{y =} \KeywordTok{expression}\NormalTok{(}\StringTok{"Discharge (ft"}\OperatorTok{^}\DecValTok{3}\OperatorTok{*}\StringTok{"/s)"}\NormalTok{))}
\KeywordTok{print}\NormalTok{(EnoDischargePlot)}
\end{Highlighting}
\end{Shaded}

\includegraphics{A06_Roth_files/figure-latex/unnamed-chunk-1-1.pdf}

\begin{Shaded}
\begin{Highlighting}[]
\NormalTok{EnoDischarge_ts <-}\StringTok{ }\KeywordTok{ts}\NormalTok{(EnoDischarge[[}\DecValTok{4}\NormalTok{]], }\DataTypeTok{frequency =} \DecValTok{365}\NormalTok{)}

\NormalTok{Eno_Decomposed <-}\StringTok{ }\KeywordTok{stl}\NormalTok{(EnoDischarge_ts, }\DataTypeTok{s.window =} \StringTok{"periodic"}\NormalTok{)}

\KeywordTok{plot}\NormalTok{(Eno_Decomposed)}
\end{Highlighting}
\end{Shaded}

\includegraphics{A06_Roth_files/figure-latex/unnamed-chunk-1-2.pdf}

\begin{Shaded}
\begin{Highlighting}[]
\NormalTok{ClearCreekDischarge_ts <-}\StringTok{ }\KeywordTok{ts}\NormalTok{(ClearCreekDischarge.Monthly[[}\DecValTok{3}\NormalTok{]], }\DataTypeTok{frequency =} \DecValTok{12}\NormalTok{)}

\NormalTok{ClearCreek_Decomposed <-}\StringTok{ }\KeywordTok{stl}\NormalTok{(ClearCreekDischarge_ts, }\DataTypeTok{s.window =} \StringTok{"periodic"}\NormalTok{)}

\KeywordTok{plot}\NormalTok{(ClearCreek_Decomposed)}
\end{Highlighting}
\end{Shaded}

\includegraphics{A06_Roth_files/figure-latex/unnamed-chunk-1-3.pdf}

\begin{enumerate}
\def\labelenumi{\arabic{enumi}.}
\setcounter{enumi}{9}
\tightlist
\item
  How do the seasonal and trend components of the decomposition compare
  to the Clear Creek discharge dataset? Are they similar in magnitude?
\end{enumerate}

\begin{quote}
Seasonal: The seasonal component has a higher magnitude for Clear Creek
has a higher magnitude than the Eno. This makes sense because Clear
Creek is monthly data and the Eno discharge data is daily.
\end{quote}

\begin{quote}
Trend: The trend component has approximately equal magnitude for both
Clear Creek and Eno, with Clear Creek's being slightly larger. Both of
them have irregular patterns with no clear directional change.
\end{quote}

\hypertarget{trend-analysis}{%
\subsection{Trend Analysis}\label{trend-analysis}}

Research question: Has there been a monotonic trend in discharge in
Clear Creek over the period of study?

\begin{enumerate}
\def\labelenumi{\arabic{enumi}.}
\setcounter{enumi}{10}
\tightlist
\item
  Generate a time series of monthly discharge in Clear Creek from the
  ClearCreekDischarge.Monthly data frame. This time series should
  include just one column (discharge).
\item
  Run a Seasonal Mann-Kendall test on the monthly discharge data.
  Inspect the overall trend and the monthly trends.
\end{enumerate}

\begin{Shaded}
\begin{Highlighting}[]
\NormalTok{ClearCreek_ts <-}\StringTok{ }\KeywordTok{ts}\NormalTok{(ClearCreekDischarge.Monthly[[}\DecValTok{3}\NormalTok{]], }\DataTypeTok{frequency =} \DecValTok{12}\NormalTok{)}

\CommentTok{# Run SMK test}
\NormalTok{ClearCreektrend <-}\StringTok{ }\KeywordTok{smk.test}\NormalTok{(ClearCreek_ts)}

\CommentTok{# Inspect results}
\NormalTok{ClearCreektrend}
\end{Highlighting}
\end{Shaded}

\begin{verbatim}
## 
##  Seasonal Mann-Kendall trend test (Hirsch-Slack test)
## 
## data:  ClearCreek_ts
## z = 1.6586, p-value = 0.09719
## alternative hypothesis: true S is not equal to 0
## sample estimates:
##      S   varS 
##    590 126102
\end{verbatim}

\begin{Shaded}
\begin{Highlighting}[]
\KeywordTok{summary}\NormalTok{(ClearCreektrend)}
\end{Highlighting}
\end{Shaded}

\begin{verbatim}
## 
##  Seasonal Mann-Kendall trend test (Hirsch-Slack test)
## 
## data: ClearCreek_ts
## alternative hypothesis: two.sided
## 
## Statistics for individual seasons
## 
## H0
##                      S  varS    tau      z Pr(>|z|)  
## Season 1:   S = 0   64 11154  0.062  0.597 0.550828  
## Season 2:   S = 0   24 10450  0.024  0.225 0.821984  
## Season 3:   S = 0   30 10450  0.030  0.284 0.776650  
## Season 4:   S = 0   35 10449  0.035  0.333 0.739425  
## Season 5:   S = 0    4 10450  0.004  0.029 0.976588  
## Season 6:   S = 0  204 10450  0.206  1.986 0.047054 *
## Season 7:   S = 0  230 10450  0.232  2.240 0.025081 *
## Season 8:   S = 0  148 10450  0.149  1.438 0.150434  
## Season 9:   S = 0   94 10450  0.095  0.910 0.362951  
## Season 10:   S = 0 -54 10450 -0.055 -0.518 0.604135  
## Season 11:   S = 0 -99 10449 -0.100 -0.959 0.337703  
## Season 12:   S = 0 -90 10450 -0.091 -0.871 0.383958  
## ---
## Signif. codes:  0 '***' 0.001 '**' 0.01 '*' 0.05 '.' 0.1 ' ' 1
\end{verbatim}

\begin{enumerate}
\def\labelenumi{\arabic{enumi}.}
\setcounter{enumi}{12}
\tightlist
\item
  Is there an overall monotonic trend in discharge over time? If so, is
  it positive or negative?
\end{enumerate}

\begin{quote}
Because the p-value is not less than 0.05, there is not a significant
overall monotonic trend in discharge over time. If the p-value would
have been below 0.05, the trend would have been positive since the
z-score was greater than 0.
\end{quote}

\begin{enumerate}
\def\labelenumi{\arabic{enumi}.}
\setcounter{enumi}{13}
\tightlist
\item
  Are there any monthly monotonic trends in discharge over time? If so,
  during which months do they occur and are they positive or negative?
\end{enumerate}

\begin{quote}
There were two monthly monotonic trends in discharge over time for the
months of June and July (p = 0.047 and p = 0.025, respectively). These
trends were both positive because they poth had z-scores greater than 0.
\end{quote}

\hypertarget{reflection}{%
\subsection{Reflection}\label{reflection}}

\begin{enumerate}
\def\labelenumi{\arabic{enumi}.}
\setcounter{enumi}{14}
\tightlist
\item
  What are 2-3 conclusions or summary points about time series you
  learned through your analysis?
\end{enumerate}

\begin{quote}
Seasonal fluctuations can vary greatly between sites and just because
there are large variations in measurements over time does not mean that
there are overall trends.
\end{quote}

\begin{enumerate}
\def\labelenumi{\arabic{enumi}.}
\setcounter{enumi}{15}
\tightlist
\item
  What data, visualizations, and/or models supported your conclusions
  from 12?
\end{enumerate}

\begin{quote}
The seasonal and trend decompositions for the Eno and Clear Creek
supported my conclusions.
\end{quote}

\begin{enumerate}
\def\labelenumi{\arabic{enumi}.}
\setcounter{enumi}{16}
\tightlist
\item
  Did hands-on data analysis impact your learning about time series
  relative to a theory-based lesson? If so, how?
\end{enumerate}

\begin{quote}
The breakdown of the seasonal mann-kendall test helped in my
understanding of time series analysis.
\end{quote}

\begin{enumerate}
\def\labelenumi{\arabic{enumi}.}
\setcounter{enumi}{17}
\tightlist
\item
  How did the real-world data compare with your expectations from
  theory?
\end{enumerate}

\begin{quote}
The real world data did not differ much from my expectations.
\end{quote}


\end{document}
