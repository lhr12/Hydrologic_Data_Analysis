\documentclass[]{article}
\usepackage{lmodern}
\usepackage{amssymb,amsmath}
\usepackage{ifxetex,ifluatex}
\usepackage{fixltx2e} % provides \textsubscript
\ifnum 0\ifxetex 1\fi\ifluatex 1\fi=0 % if pdftex
  \usepackage[T1]{fontenc}
  \usepackage[utf8]{inputenc}
\else % if luatex or xelatex
  \ifxetex
    \usepackage{mathspec}
  \else
    \usepackage{fontspec}
  \fi
  \defaultfontfeatures{Ligatures=TeX,Scale=MatchLowercase}
\fi
% use upquote if available, for straight quotes in verbatim environments
\IfFileExists{upquote.sty}{\usepackage{upquote}}{}
% use microtype if available
\IfFileExists{microtype.sty}{%
\usepackage{microtype}
\UseMicrotypeSet[protrusion]{basicmath} % disable protrusion for tt fonts
}{}
\usepackage[margin=2.54cm]{geometry}
\usepackage{hyperref}
\hypersetup{unicode=true,
            pdftitle={Assignment 2: Physical Properties of Lakes},
            pdfauthor={Lindsay Roth},
            pdfborder={0 0 0},
            breaklinks=true}
\urlstyle{same}  % don't use monospace font for urls
\usepackage{color}
\usepackage{fancyvrb}
\newcommand{\VerbBar}{|}
\newcommand{\VERB}{\Verb[commandchars=\\\{\}]}
\DefineVerbatimEnvironment{Highlighting}{Verbatim}{commandchars=\\\{\}}
% Add ',fontsize=\small' for more characters per line
\usepackage{framed}
\definecolor{shadecolor}{RGB}{248,248,248}
\newenvironment{Shaded}{\begin{snugshade}}{\end{snugshade}}
\newcommand{\AlertTok}[1]{\textcolor[rgb]{0.94,0.16,0.16}{#1}}
\newcommand{\AnnotationTok}[1]{\textcolor[rgb]{0.56,0.35,0.01}{\textbf{\textit{#1}}}}
\newcommand{\AttributeTok}[1]{\textcolor[rgb]{0.77,0.63,0.00}{#1}}
\newcommand{\BaseNTok}[1]{\textcolor[rgb]{0.00,0.00,0.81}{#1}}
\newcommand{\BuiltInTok}[1]{#1}
\newcommand{\CharTok}[1]{\textcolor[rgb]{0.31,0.60,0.02}{#1}}
\newcommand{\CommentTok}[1]{\textcolor[rgb]{0.56,0.35,0.01}{\textit{#1}}}
\newcommand{\CommentVarTok}[1]{\textcolor[rgb]{0.56,0.35,0.01}{\textbf{\textit{#1}}}}
\newcommand{\ConstantTok}[1]{\textcolor[rgb]{0.00,0.00,0.00}{#1}}
\newcommand{\ControlFlowTok}[1]{\textcolor[rgb]{0.13,0.29,0.53}{\textbf{#1}}}
\newcommand{\DataTypeTok}[1]{\textcolor[rgb]{0.13,0.29,0.53}{#1}}
\newcommand{\DecValTok}[1]{\textcolor[rgb]{0.00,0.00,0.81}{#1}}
\newcommand{\DocumentationTok}[1]{\textcolor[rgb]{0.56,0.35,0.01}{\textbf{\textit{#1}}}}
\newcommand{\ErrorTok}[1]{\textcolor[rgb]{0.64,0.00,0.00}{\textbf{#1}}}
\newcommand{\ExtensionTok}[1]{#1}
\newcommand{\FloatTok}[1]{\textcolor[rgb]{0.00,0.00,0.81}{#1}}
\newcommand{\FunctionTok}[1]{\textcolor[rgb]{0.00,0.00,0.00}{#1}}
\newcommand{\ImportTok}[1]{#1}
\newcommand{\InformationTok}[1]{\textcolor[rgb]{0.56,0.35,0.01}{\textbf{\textit{#1}}}}
\newcommand{\KeywordTok}[1]{\textcolor[rgb]{0.13,0.29,0.53}{\textbf{#1}}}
\newcommand{\NormalTok}[1]{#1}
\newcommand{\OperatorTok}[1]{\textcolor[rgb]{0.81,0.36,0.00}{\textbf{#1}}}
\newcommand{\OtherTok}[1]{\textcolor[rgb]{0.56,0.35,0.01}{#1}}
\newcommand{\PreprocessorTok}[1]{\textcolor[rgb]{0.56,0.35,0.01}{\textit{#1}}}
\newcommand{\RegionMarkerTok}[1]{#1}
\newcommand{\SpecialCharTok}[1]{\textcolor[rgb]{0.00,0.00,0.00}{#1}}
\newcommand{\SpecialStringTok}[1]{\textcolor[rgb]{0.31,0.60,0.02}{#1}}
\newcommand{\StringTok}[1]{\textcolor[rgb]{0.31,0.60,0.02}{#1}}
\newcommand{\VariableTok}[1]{\textcolor[rgb]{0.00,0.00,0.00}{#1}}
\newcommand{\VerbatimStringTok}[1]{\textcolor[rgb]{0.31,0.60,0.02}{#1}}
\newcommand{\WarningTok}[1]{\textcolor[rgb]{0.56,0.35,0.01}{\textbf{\textit{#1}}}}
\usepackage{graphicx,grffile}
\makeatletter
\def\maxwidth{\ifdim\Gin@nat@width>\linewidth\linewidth\else\Gin@nat@width\fi}
\def\maxheight{\ifdim\Gin@nat@height>\textheight\textheight\else\Gin@nat@height\fi}
\makeatother
% Scale images if necessary, so that they will not overflow the page
% margins by default, and it is still possible to overwrite the defaults
% using explicit options in \includegraphics[width, height, ...]{}
\setkeys{Gin}{width=\maxwidth,height=\maxheight,keepaspectratio}
\IfFileExists{parskip.sty}{%
\usepackage{parskip}
}{% else
\setlength{\parindent}{0pt}
\setlength{\parskip}{6pt plus 2pt minus 1pt}
}
\setlength{\emergencystretch}{3em}  % prevent overfull lines
\providecommand{\tightlist}{%
  \setlength{\itemsep}{0pt}\setlength{\parskip}{0pt}}
\setcounter{secnumdepth}{0}
% Redefines (sub)paragraphs to behave more like sections
\ifx\paragraph\undefined\else
\let\oldparagraph\paragraph
\renewcommand{\paragraph}[1]{\oldparagraph{#1}\mbox{}}
\fi
\ifx\subparagraph\undefined\else
\let\oldsubparagraph\subparagraph
\renewcommand{\subparagraph}[1]{\oldsubparagraph{#1}\mbox{}}
\fi

%%% Use protect on footnotes to avoid problems with footnotes in titles
\let\rmarkdownfootnote\footnote%
\def\footnote{\protect\rmarkdownfootnote}

%%% Change title format to be more compact
\usepackage{titling}

% Create subtitle command for use in maketitle
\providecommand{\subtitle}[1]{
  \posttitle{
    \begin{center}\large#1\end{center}
    }
}

\setlength{\droptitle}{-2em}

  \title{Assignment 2: Physical Properties of Lakes}
    \pretitle{\vspace{\droptitle}\centering\huge}
  \posttitle{\par}
    \author{Lindsay Roth}
    \preauthor{\centering\large\emph}
  \postauthor{\par}
    \date{}
    \predate{}\postdate{}
  

\begin{document}
\maketitle

\hypertarget{overview}{%
\subsection{OVERVIEW}\label{overview}}

This exercise accompanies the lessons in Hydrologic Data Analysis on the
physical properties of lakes.

\hypertarget{directions}{%
\subsection{Directions}\label{directions}}

\begin{enumerate}
\def\labelenumi{\arabic{enumi}.}
\tightlist
\item
  Change ``Student Name'' on line 3 (above) with your name.
\item
  Work through the steps, \textbf{creating code and output} that fulfill
  each instruction.
\item
  Be sure to \textbf{answer the questions} in this assignment document.
\item
  When you have completed the assignment, \textbf{Knit} the text and
  code into a single PDF file.
\item
  After Knitting, submit the completed exercise (PDF file) to the
  dropbox in Sakai. Add your last name into the file name (e.g.,
  ``Salk\_A02\_LakePhysical.Rmd'') prior to submission.
\end{enumerate}

The completed exercise is due on 11 September 2019 at 9:00 am.

\hypertarget{setup}{%
\subsection{Setup}\label{setup}}

\begin{enumerate}
\def\labelenumi{\arabic{enumi}.}
\tightlist
\item
  Verify your working directory is set to the R project file,
\item
  Load the tidyverse, lubridate, and cowplot packages
\item
  Import the NTL-LTER physical lake dataset and set the date column to
  the date format
\item
  Set your ggplot theme (can be theme\_classic or something else)
\end{enumerate}

\begin{Shaded}
\begin{Highlighting}[]
\KeywordTok{getwd}\NormalTok{()}
\end{Highlighting}
\end{Shaded}

\begin{verbatim}
## [1] "/Users/lindsayroth/Documents/MEM 2nd Year/HydroData/Hydrologic_Data_Analysis"
\end{verbatim}

\begin{Shaded}
\begin{Highlighting}[]
\KeywordTok{library}\NormalTok{(tidyverse)}
\end{Highlighting}
\end{Shaded}

\begin{verbatim}
## -- Attaching packages ----------------------------------------------------------- tidyverse 1.2.1 --
\end{verbatim}

\begin{verbatim}
## v ggplot2 3.2.1     v purrr   0.3.2
## v tibble  2.1.3     v dplyr   0.8.3
## v tidyr   0.8.3     v stringr 1.4.0
## v readr   1.3.1     v forcats 0.4.0
\end{verbatim}

\begin{verbatim}
## -- Conflicts -------------------------------------------------------------- tidyverse_conflicts() --
## x dplyr::filter() masks stats::filter()
## x dplyr::lag()    masks stats::lag()
\end{verbatim}

\begin{Shaded}
\begin{Highlighting}[]
\KeywordTok{library}\NormalTok{(lubridate)}
\end{Highlighting}
\end{Shaded}

\begin{verbatim}
## 
## Attaching package: 'lubridate'
\end{verbatim}

\begin{verbatim}
## The following object is masked from 'package:base':
## 
##     date
\end{verbatim}

\begin{Shaded}
\begin{Highlighting}[]
\KeywordTok{library}\NormalTok{(cowplot)}
\end{Highlighting}
\end{Shaded}

\begin{verbatim}
## 
## ********************************************************
\end{verbatim}

\begin{verbatim}
## Note: As of version 1.0.0, cowplot does not change the
\end{verbatim}

\begin{verbatim}
##   default ggplot2 theme anymore. To recover the previous
\end{verbatim}

\begin{verbatim}
##   behavior, execute:
##   theme_set(theme_cowplot())
\end{verbatim}

\begin{verbatim}
## ********************************************************
\end{verbatim}

\begin{verbatim}
## 
## Attaching package: 'cowplot'
\end{verbatim}

\begin{verbatim}
## The following object is masked from 'package:lubridate':
## 
##     stamp
\end{verbatim}

\begin{Shaded}
\begin{Highlighting}[]
\CommentTok{#update.packages(c("knitr", "stringr", "stringi"))}

\NormalTok{NTLdata <-}\StringTok{ }\KeywordTok{read.csv}\NormalTok{(}\StringTok{"./Data/Raw/NTL-LTER_Lake_ChemistryPhysics_Raw.csv"}\NormalTok{)}

\KeywordTok{theme_set}\NormalTok{(}\KeywordTok{theme_classic}\NormalTok{())}
\end{Highlighting}
\end{Shaded}

\hypertarget{creating-and-analyzing-lake-temperature-profiles}{%
\subsection{Creating and analyzing lake temperature
profiles}\label{creating-and-analyzing-lake-temperature-profiles}}

\hypertarget{single-lake-multiple-dates}{%
\subsubsection{Single lake, multiple
dates}\label{single-lake-multiple-dates}}

\begin{enumerate}
\def\labelenumi{\arabic{enumi}.}
\setcounter{enumi}{4}
\tightlist
\item
  Choose either Peter or Tuesday Lake. Create a new data frame that
  wrangles the full data frame so that it only includes that lake during
  two different years (one year from the early part of the dataset and
  one year from the late part of the dataset).
\end{enumerate}

\begin{Shaded}
\begin{Highlighting}[]
\NormalTok{NTLdata}\OperatorTok{$}\NormalTok{sampledate <-}\StringTok{ }\KeywordTok{as.Date}\NormalTok{(NTLdata}\OperatorTok{$}\NormalTok{sampledate, }\StringTok{"%m/%d/%y"}\NormalTok{)}

\NormalTok{TuesdayData <-}\StringTok{ }\NormalTok{NTLdata }\OperatorTok
\StringTok{  }\KeywordTok{filter}\NormalTok{(lakename }\OperatorTok{==}\StringTok{ "Tuesday Lake"}\NormalTok{)}

\NormalTok{Tuesday1985 <-}\StringTok{ }\NormalTok{TuesdayData }\OperatorTok
\StringTok{  }\KeywordTok{filter}\NormalTok{(year4 }\OperatorTok{==}\StringTok{ }\DecValTok{1985}\NormalTok{)}

\NormalTok{Tuesday2015 <-}\StringTok{ }\NormalTok{TuesdayData }\OperatorTok
\StringTok{  }\KeywordTok{filter}\NormalTok{(year4 }\OperatorTok{==}\StringTok{ }\DecValTok{2015}\NormalTok{)}
\end{Highlighting}
\end{Shaded}

\begin{enumerate}
\def\labelenumi{\arabic{enumi}.}
\setcounter{enumi}{5}
\tightlist
\item
  Create three graphs: (1) temperature profiles for the early year, (2)
  temperature profiles for the late year, and (3) a \texttt{plot\_grid}
  of the two graphs together. Choose \texttt{geom\_point} and color your
  points by date.
\end{enumerate}

Remember to edit your graphs so they follow good data visualization
practices.

\begin{Shaded}
\begin{Highlighting}[]
\NormalTok{Tempprofiles1985 <-}\StringTok{ }
\StringTok{  }\KeywordTok{ggplot}\NormalTok{(Tuesday1985, }\KeywordTok{aes}\NormalTok{(}\DataTypeTok{x =}\NormalTok{ temperature_C, }\DataTypeTok{y =}\NormalTok{ depth, }\DataTypeTok{color =}\NormalTok{ daynum)) }\OperatorTok{+}
\StringTok{  }\KeywordTok{geom_point}\NormalTok{() }\OperatorTok{+}
\StringTok{  }\KeywordTok{scale_y_reverse}\NormalTok{() }\OperatorTok{+}
\StringTok{  }\KeywordTok{scale_x_continuous}\NormalTok{(}\DataTypeTok{position =} \StringTok{"top"}\NormalTok{) }\OperatorTok{+}
\StringTok{  }\KeywordTok{scale_color_viridis_c}\NormalTok{(}\DataTypeTok{end =} \FloatTok{0.8}\NormalTok{) }\OperatorTok{+}\StringTok{ }
\StringTok{  }\CommentTok{#0.8 telling viridis not to use yellow}
\StringTok{  }\KeywordTok{labs}\NormalTok{(}\DataTypeTok{x =} \KeywordTok{expression}\NormalTok{(}\StringTok{"Temperature "}\NormalTok{(degree}\OperatorTok{*}\NormalTok{C)), }\DataTypeTok{y =} \StringTok{"Depth (m)"}\NormalTok{) }\OperatorTok{+}
\StringTok{  }\KeywordTok{theme}\NormalTok{(}\DataTypeTok{legend.position =} \StringTok{"none"}\NormalTok{)}
\KeywordTok{print}\NormalTok{(Tempprofiles1985)}
\end{Highlighting}
\end{Shaded}

\begin{verbatim}
## Warning: Removed 36 rows containing missing values (geom_point).
\end{verbatim}

\includegraphics{A02_Roth_files/figure-latex/unnamed-chunk-3-1.pdf}

\begin{Shaded}
\begin{Highlighting}[]
\NormalTok{Tempprofiles2015 <-}\StringTok{ }
\StringTok{  }\KeywordTok{ggplot}\NormalTok{(Tuesday2015, }\KeywordTok{aes}\NormalTok{(}\DataTypeTok{x =}\NormalTok{ temperature_C, }\DataTypeTok{y =}\NormalTok{ depth, }\DataTypeTok{color =}\NormalTok{ daynum)) }\OperatorTok{+}
\StringTok{  }\KeywordTok{geom_point}\NormalTok{() }\OperatorTok{+}
\StringTok{  }\KeywordTok{scale_y_reverse}\NormalTok{() }\OperatorTok{+}
\StringTok{  }\KeywordTok{scale_x_continuous}\NormalTok{(}\DataTypeTok{position =} \StringTok{"top"}\NormalTok{) }\OperatorTok{+}
\StringTok{  }\KeywordTok{scale_color_viridis_c}\NormalTok{(}\DataTypeTok{end =} \FloatTok{0.8}\NormalTok{) }\OperatorTok{+}\StringTok{ }
\StringTok{  }\CommentTok{#0.8 telling viridis not to use yellow}
\StringTok{  }\KeywordTok{labs}\NormalTok{(}\DataTypeTok{x =} \KeywordTok{expression}\NormalTok{(}\StringTok{"Temperature "}\NormalTok{(degree}\OperatorTok{*}\NormalTok{C)), }\DataTypeTok{y =} \StringTok{"Depth (m)"}\NormalTok{) }\OperatorTok{+}
\StringTok{  }\KeywordTok{theme}\NormalTok{(}\DataTypeTok{axis.text.y =} \KeywordTok{element_blank}\NormalTok{(), }\DataTypeTok{axis.title.y =} \KeywordTok{element_blank}\NormalTok{())}
\KeywordTok{print}\NormalTok{(Tempprofiles2015)}
\end{Highlighting}
\end{Shaded}

\begin{verbatim}
## Warning: Removed 30 rows containing missing values (geom_point).
\end{verbatim}

\includegraphics{A02_Roth_files/figure-latex/unnamed-chunk-3-2.pdf}

\begin{Shaded}
\begin{Highlighting}[]
\NormalTok{TuesdayTempProfiles <-}\StringTok{ }
\StringTok{  }\KeywordTok{plot_grid}\NormalTok{(Tempprofiles1985, Tempprofiles2015, }
            \DataTypeTok{ncol =} \DecValTok{2}\NormalTok{,}
            \DataTypeTok{rel_widths =} \KeywordTok{c}\NormalTok{(}\DecValTok{1}\NormalTok{, }\FloatTok{1.25}\NormalTok{))}
\end{Highlighting}
\end{Shaded}

\begin{verbatim}
## Warning: Removed 36 rows containing missing values (geom_point).
\end{verbatim}

\begin{verbatim}
## Warning: Removed 30 rows containing missing values (geom_point).
\end{verbatim}

\begin{Shaded}
\begin{Highlighting}[]
\KeywordTok{print}\NormalTok{(TuesdayTempProfiles)}
\end{Highlighting}
\end{Shaded}

\includegraphics{A02_Roth_files/figure-latex/unnamed-chunk-3-3.pdf}

\begin{enumerate}
\def\labelenumi{\arabic{enumi}.}
\setcounter{enumi}{6}
\tightlist
\item
  Interpret the stratification patterns in your graphs in light of
  seasonal trends. In addition, do you see differences between the two
  years?
\end{enumerate}

\begin{quote}
For the most part, 1985 and 2015 are similar. The warmest months are
around the month of July, and the coolest monts are May and August.
However, there are a couple of noticable differences. The temperature
range in the epilimnion is greater in 2015 than it is in 1985. There is
deeper mixing of surface temperatures in 2015, resulting in a deeper
epilimnion in 2015. The thermocline in 2015 also appears to extent to
deeper depths than 1985.
\end{quote}

\hypertarget{multiple-lakes-single-date}{%
\subsubsection{Multiple lakes, single
date}\label{multiple-lakes-single-date}}

\begin{enumerate}
\def\labelenumi{\arabic{enumi}.}
\setcounter{enumi}{7}
\tightlist
\item
  On July 25, 26, and 27 in 2016, all three lakes (Peter, Paul, and
  Tuesday) were sampled. Wrangle your data frame to include just these
  three dates.
\end{enumerate}

\begin{Shaded}
\begin{Highlighting}[]
\NormalTok{JulyData <-}\StringTok{ }\NormalTok{NTLdata }\OperatorTok
\StringTok{  }\KeywordTok{filter}\NormalTok{(lakename }\OperatorTok{==}\StringTok{ "Peter Lake"} \OperatorTok{|}\StringTok{ }\NormalTok{lakename }\OperatorTok{==}\StringTok{ "Paul Lake"} \OperatorTok{|}
\StringTok{           }\NormalTok{lakename }\OperatorTok{==}\StringTok{ "Tuesday Lake"}\NormalTok{) }\OperatorTok
\StringTok{  }\KeywordTok{filter}\NormalTok{(sampledate }\OperatorTok{==}\StringTok{ }\DecValTok{2016-07-25} \OperatorTok{|}\StringTok{ }\NormalTok{sampledate }\OperatorTok{==}\StringTok{ }\DecValTok{2016-07-26} \OperatorTok{|}
\StringTok{           }\NormalTok{sampledate }\OperatorTok{==}\StringTok{ }\DecValTok{2016-07-27}\NormalTok{)}
\end{Highlighting}
\end{Shaded}

\begin{enumerate}
\def\labelenumi{\arabic{enumi}.}
\setcounter{enumi}{8}
\tightlist
\item
  Plot a profile line graph of temperature by depth, one line per lake.
  Each lake can be designated by a separate color.
\end{enumerate}

\begin{Shaded}
\begin{Highlighting}[]
\KeywordTok{as.numeric}\NormalTok{(JulyData}\OperatorTok{$}\NormalTok{temperature_C)}
\end{Highlighting}
\end{Shaded}

\begin{verbatim}
## numeric(0)
\end{verbatim}

\begin{Shaded}
\begin{Highlighting}[]
\NormalTok{TempprofilesJuly<-}\StringTok{ }
\StringTok{  }\KeywordTok{ggplot}\NormalTok{(JulyData, }\KeywordTok{aes}\NormalTok{(}\DataTypeTok{x =}\NormalTok{ temperature_C, }\DataTypeTok{y =}\NormalTok{ depth, }\DataTypeTok{color =}\NormalTok{ lakename)) }\OperatorTok{+}
\StringTok{  }\KeywordTok{geom_point}\NormalTok{() }\OperatorTok{+}
\StringTok{  }\KeywordTok{scale_y_reverse}\NormalTok{() }\OperatorTok{+}
\StringTok{  }\KeywordTok{scale_x_continuous}\NormalTok{(}\DataTypeTok{position =} \StringTok{"top"}\NormalTok{) }\OperatorTok{+}
\StringTok{  }\KeywordTok{scale_color_viridis_d}\NormalTok{(}\DataTypeTok{end =} \FloatTok{0.8}\NormalTok{) }\OperatorTok{+}\StringTok{ }
\StringTok{  }\KeywordTok{labs}\NormalTok{(}\DataTypeTok{x =} \KeywordTok{expression}\NormalTok{(}\StringTok{"Temperature "}\NormalTok{(degree}\OperatorTok{*}\NormalTok{C)), }\DataTypeTok{y =} \StringTok{"Depth (m)"}\NormalTok{)}
\KeywordTok{print}\NormalTok{(TempprofilesJuly)}
\end{Highlighting}
\end{Shaded}

\includegraphics{A02_Roth_files/figure-latex/unnamed-chunk-5-1.pdf}

\begin{enumerate}
\def\labelenumi{\arabic{enumi}.}
\setcounter{enumi}{9}
\tightlist
\item
  What is the depth range of the epilimnion in each lake? The
  thermocline? The hypolimnion?
\end{enumerate}

\begin{quote}
Paul Lake: The epilimnion is approximately 1.5-2 m deep, the thermocline
from 2-6 m, and the hypolimnion is 6 m and lower. Peter Lake: The
epilimnion is approximately 2-2.5 m deep, the thermocline from 2.5-7.5
m, and the hypolimnion 7.5 m and lower. Tuesday Lake: The epilimnion is
about 2.5 m deep, the thermocline 2.5-8 m, and the hypolimnion 8 m and
deeper.
\end{quote}

\hypertarget{trends-in-surface-temperatures-over-time.}{%
\subsection{Trends in surface temperatures over
time.}\label{trends-in-surface-temperatures-over-time.}}

\begin{enumerate}
\def\labelenumi{\arabic{enumi}.}
\setcounter{enumi}{10}
\tightlist
\item
  Run the same analyses we ran in class to determine if surface lake
  temperatures for a given month have increased over time (``Long-term
  change in temperature'' section of day 4 lesson in its entirety), this
  time for either Peter or Tuesday Lake.
\end{enumerate}

\begin{Shaded}
\begin{Highlighting}[]
\NormalTok{Tuesdaydata.surface <-}\StringTok{ }\NormalTok{TuesdayData }\OperatorTok
\StringTok{  }\KeywordTok{mutate}\NormalTok{(}\DataTypeTok{Month =} \KeywordTok{month}\NormalTok{(sampledate)) }\OperatorTok
\StringTok{  }\KeywordTok{filter}\NormalTok{(Month }\OperatorTok{==}\StringTok{ }\DecValTok{5} \OperatorTok{|}\StringTok{ }\NormalTok{Month }\OperatorTok{==}\StringTok{ }\DecValTok{6} \OperatorTok{|}\StringTok{ }\NormalTok{Month }\OperatorTok{==}\StringTok{ }\DecValTok{7} \OperatorTok{|}\StringTok{ }\NormalTok{Month }\OperatorTok{==}\StringTok{ }\DecValTok{8}\NormalTok{) }\OperatorTok
\StringTok{  }\KeywordTok{filter}\NormalTok{(depth }\OperatorTok{==}\StringTok{ }\FloatTok{0.00}\NormalTok{)}

\NormalTok{Tuesdaydata.May <-}\StringTok{ }\NormalTok{Tuesdaydata.surface }\OperatorTok
\StringTok{  }\KeywordTok{filter}\NormalTok{(Month }\OperatorTok{==}\StringTok{ }\DecValTok{5}\NormalTok{)}

\NormalTok{Tuesdaydata.Jun <-}\StringTok{ }\NormalTok{Tuesdaydata.surface }\OperatorTok
\StringTok{  }\KeywordTok{filter}\NormalTok{(Month }\OperatorTok{==}\StringTok{ }\DecValTok{6}\NormalTok{)}

\NormalTok{Tuesdaydata.Jul <-}\StringTok{ }\NormalTok{Tuesdaydata.surface }\OperatorTok
\StringTok{  }\KeywordTok{filter}\NormalTok{(Month }\OperatorTok{==}\StringTok{ }\DecValTok{7}\NormalTok{)}

\NormalTok{Tuesdaydata.Aug <-}\StringTok{ }\NormalTok{Tuesdaydata.surface }\OperatorTok
\StringTok{  }\KeywordTok{filter}\NormalTok{(Month }\OperatorTok{==}\StringTok{ }\DecValTok{8}\NormalTok{)}

\CommentTok{#Step 4}
\NormalTok{May.Test <-}\StringTok{ }\KeywordTok{lm}\NormalTok{(temperature_C }\OperatorTok{~}\StringTok{ }\NormalTok{year4, Tuesdaydata.May)}
\KeywordTok{summary}\NormalTok{(May.Test)}
\end{Highlighting}
\end{Shaded}

\begin{verbatim}
## 
## Call:
## lm(formula = temperature_C ~ year4, data = Tuesdaydata.May)
## 
## Residuals:
##     Min      1Q  Median      3Q     Max 
## -4.6223 -1.4411  0.0314  1.5604  5.2216 
## 
## Coefficients:
##              Estimate Std. Error t value Pr(>|t|)
## (Intercept) -27.15303   73.73032  -0.368    0.715
## year4         0.02196    0.03689   0.595    0.556
## 
## Residual standard error: 2.522 on 32 degrees of freedom
## Multiple R-squared:  0.01095,    Adjusted R-squared:  -0.01995 
## F-statistic: 0.3544 on 1 and 32 DF,  p-value: 0.5558
\end{verbatim}

\begin{Shaded}
\begin{Highlighting}[]
\NormalTok{Jun.Test <-}\StringTok{ }\KeywordTok{lm}\NormalTok{(temperature_C }\OperatorTok{~}\StringTok{ }\NormalTok{year4, Tuesdaydata.Jun)}
\KeywordTok{summary}\NormalTok{(Jun.Test)}
\end{Highlighting}
\end{Shaded}

\begin{verbatim}
## 
## Call:
## lm(formula = temperature_C ~ year4, data = Tuesdaydata.Jun)
## 
## Residuals:
##     Min      1Q  Median      3Q     Max 
## -6.0339 -1.5343 -0.0279  1.9180  6.7676 
## 
## Coefficients:
##               Estimate Std. Error t value Pr(>|t|)
## (Intercept) 21.1373460 50.7897026   0.416    0.678
## year4       -0.0002531  0.0254253  -0.010    0.992
## 
## Residual standard error: 2.621 on 80 degrees of freedom
## Multiple R-squared:  1.239e-06,  Adjusted R-squared:  -0.0125 
## F-statistic: 9.912e-05 on 1 and 80 DF,  p-value: 0.9921
\end{verbatim}

\begin{Shaded}
\begin{Highlighting}[]
\NormalTok{Jul.Test <-}\StringTok{ }\KeywordTok{lm}\NormalTok{(temperature_C }\OperatorTok{~}\StringTok{ }\NormalTok{year4, Tuesdaydata.Jul)}
\KeywordTok{summary}\NormalTok{(Jul.Test)}
\end{Highlighting}
\end{Shaded}

\begin{verbatim}
## 
## Call:
## lm(formula = temperature_C ~ year4, data = Tuesdaydata.Jul)
## 
## Residuals:
##     Min      1Q  Median      3Q     Max 
## -4.0561 -1.3275 -0.2047  1.4031  4.2161 
## 
## Coefficients:
##              Estimate Std. Error t value Pr(>|t|)  
## (Intercept) -49.18776   37.36614  -1.316   0.1916  
## year4         0.03612    0.01871   1.931   0.0569 .
## ---
## Signif. codes:  0 '***' 0.001 '**' 0.01 '*' 0.05 '.' 0.1 ' ' 1
## 
## Residual standard error: 1.953 on 84 degrees of freedom
##   (1 observation deleted due to missingness)
## Multiple R-squared:  0.04248,    Adjusted R-squared:  0.03109 
## F-statistic: 3.727 on 1 and 84 DF,  p-value: 0.05691
\end{verbatim}

\begin{Shaded}
\begin{Highlighting}[]
\NormalTok{Aug.Test <-}\StringTok{ }\KeywordTok{lm}\NormalTok{(temperature_C }\OperatorTok{~}\StringTok{ }\NormalTok{year4, Tuesdaydata.Aug)}
\KeywordTok{summary}\NormalTok{(Aug.Test)}
\end{Highlighting}
\end{Shaded}

\begin{verbatim}
## 
## Call:
## lm(formula = temperature_C ~ year4, data = Tuesdaydata.Aug)
## 
## Residuals:
##     Min      1Q  Median      3Q     Max 
## -4.9656 -1.1055 -0.0787  1.2820  3.8677 
## 
## Coefficients:
##              Estimate Std. Error t value Pr(>|t|)
## (Intercept) -37.70343   41.36954  -0.911    0.365
## year4         0.02976    0.02072   1.436    0.155
## 
## Residual standard error: 2.025 on 81 degrees of freedom
## Multiple R-squared:  0.02484,    Adjusted R-squared:  0.0128 
## F-statistic: 2.063 on 1 and 81 DF,  p-value: 0.1547
\end{verbatim}

\begin{Shaded}
\begin{Highlighting}[]
\NormalTok{TempChangePlot <-}\StringTok{ }
\StringTok{  }\KeywordTok{ggplot}\NormalTok{(Tuesdaydata.surface, }\KeywordTok{aes}\NormalTok{(}\DataTypeTok{x =}\NormalTok{ sampledate, }\DataTypeTok{y =}\NormalTok{ temperature_C)) }\OperatorTok{+}
\StringTok{  }\KeywordTok{geom_point}\NormalTok{() }\OperatorTok{+}
\StringTok{  }\KeywordTok{geom_smooth}\NormalTok{(}\DataTypeTok{se =} \OtherTok{FALSE}\NormalTok{, }\DataTypeTok{method =}\NormalTok{ lm) }\OperatorTok{+}
\StringTok{  }\KeywordTok{facet_grid}\NormalTok{(}\DataTypeTok{rows =} \KeywordTok{vars}\NormalTok{(Month))}
\KeywordTok{print}\NormalTok{(TempChangePlot)}
\end{Highlighting}
\end{Shaded}

\begin{verbatim}
## Warning: Removed 1 rows containing non-finite values (stat_smooth).
\end{verbatim}

\begin{verbatim}
## Warning: Removed 1 rows containing missing values (geom_point).
\end{verbatim}

\includegraphics{A02_Roth_files/figure-latex/unnamed-chunk-6-1.pdf}

\begin{enumerate}
\def\labelenumi{\arabic{enumi}.}
\setcounter{enumi}{11}
\tightlist
\item
  How do your results compare to those we found in class for Paul Lake?
  Do similar trends exist for both lakes?
\end{enumerate}

\begin{quote}
There are some similar trends between the Paul Lake and Tuesday Lake
tests. The largest coefficients were for the months of July and August,
and the smallest coefficients were for the months of May and June.
However, for Paul Lake the tests for July and August were statistically
significant, while none of the tests for Tuesday Lake were statistically
significant.
\end{quote}


\end{document}
