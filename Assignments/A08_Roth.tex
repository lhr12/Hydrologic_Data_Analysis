\documentclass[]{article}
\usepackage{lmodern}
\usepackage{amssymb,amsmath}
\usepackage{ifxetex,ifluatex}
\usepackage{fixltx2e} % provides \textsubscript
\ifnum 0\ifxetex 1\fi\ifluatex 1\fi=0 % if pdftex
  \usepackage[T1]{fontenc}
  \usepackage[utf8]{inputenc}
\else % if luatex or xelatex
  \ifxetex
    \usepackage{mathspec}
  \else
    \usepackage{fontspec}
  \fi
  \defaultfontfeatures{Ligatures=TeX,Scale=MatchLowercase}
\fi
% use upquote if available, for straight quotes in verbatim environments
\IfFileExists{upquote.sty}{\usepackage{upquote}}{}
% use microtype if available
\IfFileExists{microtype.sty}{%
\usepackage{microtype}
\UseMicrotypeSet[protrusion]{basicmath} % disable protrusion for tt fonts
}{}
\usepackage[margin=2.54cm]{geometry}
\usepackage{hyperref}
\hypersetup{unicode=true,
            pdftitle={Assignment 8: Mapping},
            pdfauthor={Lindsay Roth},
            pdfborder={0 0 0},
            breaklinks=true}
\urlstyle{same}  % don't use monospace font for urls
\usepackage{color}
\usepackage{fancyvrb}
\newcommand{\VerbBar}{|}
\newcommand{\VERB}{\Verb[commandchars=\\\{\}]}
\DefineVerbatimEnvironment{Highlighting}{Verbatim}{commandchars=\\\{\}}
% Add ',fontsize=\small' for more characters per line
\usepackage{framed}
\definecolor{shadecolor}{RGB}{248,248,248}
\newenvironment{Shaded}{\begin{snugshade}}{\end{snugshade}}
\newcommand{\AlertTok}[1]{\textcolor[rgb]{0.94,0.16,0.16}{#1}}
\newcommand{\AnnotationTok}[1]{\textcolor[rgb]{0.56,0.35,0.01}{\textbf{\textit{#1}}}}
\newcommand{\AttributeTok}[1]{\textcolor[rgb]{0.77,0.63,0.00}{#1}}
\newcommand{\BaseNTok}[1]{\textcolor[rgb]{0.00,0.00,0.81}{#1}}
\newcommand{\BuiltInTok}[1]{#1}
\newcommand{\CharTok}[1]{\textcolor[rgb]{0.31,0.60,0.02}{#1}}
\newcommand{\CommentTok}[1]{\textcolor[rgb]{0.56,0.35,0.01}{\textit{#1}}}
\newcommand{\CommentVarTok}[1]{\textcolor[rgb]{0.56,0.35,0.01}{\textbf{\textit{#1}}}}
\newcommand{\ConstantTok}[1]{\textcolor[rgb]{0.00,0.00,0.00}{#1}}
\newcommand{\ControlFlowTok}[1]{\textcolor[rgb]{0.13,0.29,0.53}{\textbf{#1}}}
\newcommand{\DataTypeTok}[1]{\textcolor[rgb]{0.13,0.29,0.53}{#1}}
\newcommand{\DecValTok}[1]{\textcolor[rgb]{0.00,0.00,0.81}{#1}}
\newcommand{\DocumentationTok}[1]{\textcolor[rgb]{0.56,0.35,0.01}{\textbf{\textit{#1}}}}
\newcommand{\ErrorTok}[1]{\textcolor[rgb]{0.64,0.00,0.00}{\textbf{#1}}}
\newcommand{\ExtensionTok}[1]{#1}
\newcommand{\FloatTok}[1]{\textcolor[rgb]{0.00,0.00,0.81}{#1}}
\newcommand{\FunctionTok}[1]{\textcolor[rgb]{0.00,0.00,0.00}{#1}}
\newcommand{\ImportTok}[1]{#1}
\newcommand{\InformationTok}[1]{\textcolor[rgb]{0.56,0.35,0.01}{\textbf{\textit{#1}}}}
\newcommand{\KeywordTok}[1]{\textcolor[rgb]{0.13,0.29,0.53}{\textbf{#1}}}
\newcommand{\NormalTok}[1]{#1}
\newcommand{\OperatorTok}[1]{\textcolor[rgb]{0.81,0.36,0.00}{\textbf{#1}}}
\newcommand{\OtherTok}[1]{\textcolor[rgb]{0.56,0.35,0.01}{#1}}
\newcommand{\PreprocessorTok}[1]{\textcolor[rgb]{0.56,0.35,0.01}{\textit{#1}}}
\newcommand{\RegionMarkerTok}[1]{#1}
\newcommand{\SpecialCharTok}[1]{\textcolor[rgb]{0.00,0.00,0.00}{#1}}
\newcommand{\SpecialStringTok}[1]{\textcolor[rgb]{0.31,0.60,0.02}{#1}}
\newcommand{\StringTok}[1]{\textcolor[rgb]{0.31,0.60,0.02}{#1}}
\newcommand{\VariableTok}[1]{\textcolor[rgb]{0.00,0.00,0.00}{#1}}
\newcommand{\VerbatimStringTok}[1]{\textcolor[rgb]{0.31,0.60,0.02}{#1}}
\newcommand{\WarningTok}[1]{\textcolor[rgb]{0.56,0.35,0.01}{\textbf{\textit{#1}}}}
\usepackage{graphicx,grffile}
\makeatletter
\def\maxwidth{\ifdim\Gin@nat@width>\linewidth\linewidth\else\Gin@nat@width\fi}
\def\maxheight{\ifdim\Gin@nat@height>\textheight\textheight\else\Gin@nat@height\fi}
\makeatother
% Scale images if necessary, so that they will not overflow the page
% margins by default, and it is still possible to overwrite the defaults
% using explicit options in \includegraphics[width, height, ...]{}
\setkeys{Gin}{width=\maxwidth,height=\maxheight,keepaspectratio}
\IfFileExists{parskip.sty}{%
\usepackage{parskip}
}{% else
\setlength{\parindent}{0pt}
\setlength{\parskip}{6pt plus 2pt minus 1pt}
}
\setlength{\emergencystretch}{3em}  % prevent overfull lines
\providecommand{\tightlist}{%
  \setlength{\itemsep}{0pt}\setlength{\parskip}{0pt}}
\setcounter{secnumdepth}{0}
% Redefines (sub)paragraphs to behave more like sections
\ifx\paragraph\undefined\else
\let\oldparagraph\paragraph
\renewcommand{\paragraph}[1]{\oldparagraph{#1}\mbox{}}
\fi
\ifx\subparagraph\undefined\else
\let\oldsubparagraph\subparagraph
\renewcommand{\subparagraph}[1]{\oldsubparagraph{#1}\mbox{}}
\fi

%%% Use protect on footnotes to avoid problems with footnotes in titles
\let\rmarkdownfootnote\footnote%
\def\footnote{\protect\rmarkdownfootnote}

%%% Change title format to be more compact
\usepackage{titling}

% Create subtitle command for use in maketitle
\providecommand{\subtitle}[1]{
  \posttitle{
    \begin{center}\large#1\end{center}
    }
}

\setlength{\droptitle}{-2em}

  \title{Assignment 8: Mapping}
    \pretitle{\vspace{\droptitle}\centering\huge}
  \posttitle{\par}
    \author{Lindsay Roth}
    \preauthor{\centering\large\emph}
  \postauthor{\par}
    \date{}
    \predate{}\postdate{}
  

\begin{document}
\maketitle

\hypertarget{overview}{%
\subsection{OVERVIEW}\label{overview}}

This exercise accompanies the lessons in Hydrologic Data Analysis on
mapping

\hypertarget{directions}{%
\subsection{Directions}\label{directions}}

\begin{enumerate}
\def\labelenumi{\arabic{enumi}.}
\tightlist
\item
  Change ``Student Name'' on line 3 (above) with your name.
\item
  Work through the steps, \textbf{creating code and output} that fulfill
  each instruction.
\item
  Be sure to \textbf{answer the questions} in this assignment document.
\item
  When you have completed the assignment, \textbf{Knit} the text and
  code into a single pdf file.
\item
  After Knitting, submit the completed exercise (pdf file) to the
  dropbox in Sakai. Add your last name into the file name (e.g.,
  ``A08\_Salk.html'') prior to submission.
\end{enumerate}

The completed exercise is due on 23 October 2019 at 9:00 am.

\hypertarget{setup}{%
\subsection{Setup}\label{setup}}

\begin{enumerate}
\def\labelenumi{\arabic{enumi}.}
\tightlist
\item
  Verify your working directory is set to the R project file,
\item
  Load the tidyverse, lubridate, cowplot, LAGOSNE, sf, maps, and viridis
  packages.
\item
  Set your ggplot theme (can be theme\_classic or something else)
\item
  Load the lagos database, the USA rivers water features shape file, and
  the HUC6 watershed shape file.
\end{enumerate}

\begin{Shaded}
\begin{Highlighting}[]
\KeywordTok{getwd}\NormalTok{()}
\end{Highlighting}
\end{Shaded}

\begin{verbatim}
## [1] "/Users/lindsayroth/Documents/MEM 2nd Year/HydroData/Hydrologic_Data_Analysis"
\end{verbatim}

\begin{Shaded}
\begin{Highlighting}[]
\KeywordTok{library}\NormalTok{(tidyverse)}
\end{Highlighting}
\end{Shaded}

\begin{verbatim}
## -- Attaching packages ------------------------------------------------------------------------ tidyverse 1.2.1 --
\end{verbatim}

\begin{verbatim}
## v ggplot2 3.2.1     v purrr   0.3.2
## v tibble  2.1.3     v dplyr   0.8.3
## v tidyr   1.0.0     v stringr 1.4.0
## v readr   1.3.1     v forcats 0.4.0
\end{verbatim}

\begin{verbatim}
## -- Conflicts --------------------------------------------------------------------------- tidyverse_conflicts() --
## x dplyr::filter() masks stats::filter()
## x dplyr::lag()    masks stats::lag()
\end{verbatim}

\begin{Shaded}
\begin{Highlighting}[]
\KeywordTok{library}\NormalTok{(lubridate)}
\end{Highlighting}
\end{Shaded}

\begin{verbatim}
## 
## Attaching package: 'lubridate'
\end{verbatim}

\begin{verbatim}
## The following object is masked from 'package:base':
## 
##     date
\end{verbatim}

\begin{Shaded}
\begin{Highlighting}[]
\KeywordTok{library}\NormalTok{(cowplot)}
\end{Highlighting}
\end{Shaded}

\begin{verbatim}
## 
## ********************************************************
\end{verbatim}

\begin{verbatim}
## Note: As of version 1.0.0, cowplot does not change the
\end{verbatim}

\begin{verbatim}
##   default ggplot2 theme anymore. To recover the previous
\end{verbatim}

\begin{verbatim}
##   behavior, execute:
##   theme_set(theme_cowplot())
\end{verbatim}

\begin{verbatim}
## ********************************************************
\end{verbatim}

\begin{verbatim}
## 
## Attaching package: 'cowplot'
\end{verbatim}

\begin{verbatim}
## The following object is masked from 'package:lubridate':
## 
##     stamp
\end{verbatim}

\begin{Shaded}
\begin{Highlighting}[]
\KeywordTok{library}\NormalTok{(LAGOSNE)}
\KeywordTok{library}\NormalTok{(sf)}
\end{Highlighting}
\end{Shaded}

\begin{verbatim}
## Linking to GEOS 3.7.2, GDAL 2.4.2, PROJ 5.2.0
\end{verbatim}

\begin{Shaded}
\begin{Highlighting}[]
\KeywordTok{library}\NormalTok{(maps)}
\end{Highlighting}
\end{Shaded}

\begin{verbatim}
## 
## Attaching package: 'maps'
\end{verbatim}

\begin{verbatim}
## The following object is masked from 'package:purrr':
## 
##     map
\end{verbatim}

\begin{Shaded}
\begin{Highlighting}[]
\KeywordTok{library}\NormalTok{(viridis)}
\end{Highlighting}
\end{Shaded}

\begin{verbatim}
## Loading required package: viridisLite
\end{verbatim}

\begin{Shaded}
\begin{Highlighting}[]
\KeywordTok{library}\NormalTok{(RColorBrewer)}

\KeywordTok{theme_set}\NormalTok{(}\KeywordTok{theme_classic}\NormalTok{())}

\NormalTok{LAGOSdata <-}\StringTok{ }\KeywordTok{lagosne_load}\NormalTok{()}
\end{Highlighting}
\end{Shaded}

\begin{verbatim}
## Warning in `_f`(version = version, fpath = fpath): LAGOSNE version
## unspecified, loading version: 1.087.3
\end{verbatim}

\begin{Shaded}
\begin{Highlighting}[]
\NormalTok{waterfeatures <-}\StringTok{ }\KeywordTok{st_read}\NormalTok{(}\StringTok{"./Data/Raw/hydrogl020.dbf"}\NormalTok{)}
\end{Highlighting}
\end{Shaded}

\begin{verbatim}
## Reading layer `hydrogl020' from data source `/Users/lindsayroth/Documents/MEM 2nd Year/HydroData/Hydrologic_Data_Analysis/Data/Raw/hydrogl020.dbf' using driver `ESRI Shapefile'
## Simple feature collection with 76975 features and 12 fields
## geometry type:  LINESTRING
## dimension:      XY
## bbox:           xmin: -179.9982 ymin: 17.67469 xmax: 179.9831 ymax: 71.39819
## epsg (SRID):    NA
## proj4string:    NA
\end{verbatim}

\begin{Shaded}
\begin{Highlighting}[]
\NormalTok{HUC6 <-}\StringTok{ }\KeywordTok{st_read}\NormalTok{(}\StringTok{"./Data/Raw/Watersheds_Spatial/WBDHU6.dbf"}\NormalTok{)}
\end{Highlighting}
\end{Shaded}

\begin{verbatim}
## Reading layer `WBDHU6' from data source `/Users/lindsayroth/Documents/MEM 2nd Year/HydroData/Hydrologic_Data_Analysis/Data/Raw/Watersheds_Spatial/WBDHU6.dbf' using driver `ESRI Shapefile'
## Simple feature collection with 33 features and 15 fields
## geometry type:  MULTIPOLYGON
## dimension:      XY
## bbox:           xmin: -90.6235 ymin: 24.39533 xmax: -75.3981 ymax: 37.52103
## epsg (SRID):    4269
## proj4string:    +proj=longlat +ellps=GRS80 +towgs84=0,0,0,0,0,0,0 +no_defs
\end{verbatim}

\hypertarget{mapping-water-quality-in-lakes}{%
\subsection{Mapping water quality in
lakes}\label{mapping-water-quality-in-lakes}}

Complete the in-class exercise from lesson 15, to map average secchi
depth measurements across states in Maine, considering lake area and
lake depth as predictors for water clarity. Steps here are identical to
the lesson, with the following edits:

\begin{itemize}
\tightlist
\item
  Make sure all your wrangling is done in this document (this includes
  basic wrangling of the LAGOS database)
\item
  In your cowplot, do not adjust the legend items (even though they look
  ugly). Rather, reflect on how you would improve them with additional
  coding.
\item
  For item 9, \textbf{do} run a regression on secchi depth by lake area
  and a separate regression on secchi depth by lake depth. Make
  scatterplots of these relationships. Note that log-transforming one of
  these items may be necessary.
\end{itemize}

\begin{enumerate}
\def\labelenumi{\arabic{enumi}.}
\setcounter{enumi}{4}
\tightlist
\item
  Filter the states and secchi depth datasets so that they contain Maine
  only. For the secchi depth dataset, create a summary dataset with just
  the mean secchi depth.
\end{enumerate}

\begin{Shaded}
\begin{Highlighting}[]
\NormalTok{states <-}\StringTok{ }\KeywordTok{st_as_sf}\NormalTok{(}\KeywordTok{map}\NormalTok{(}\DataTypeTok{database =} \StringTok{"state"}\NormalTok{, }\DataTypeTok{plot =} \OtherTok{TRUE}\NormalTok{, }\DataTypeTok{fill =} \OtherTok{TRUE}\NormalTok{, }\DataTypeTok{col =} \StringTok{"white"}\NormalTok{))}
\end{Highlighting}
\end{Shaded}

\includegraphics{A08_Roth_files/figure-latex/unnamed-chunk-1-1.pdf}

\begin{Shaded}
\begin{Highlighting}[]
\NormalTok{LAGOSlocus <-}\StringTok{ }\NormalTok{LAGOSdata}\OperatorTok{$}\NormalTok{locus}
\NormalTok{LAGOSstate <-}\StringTok{ }\NormalTok{LAGOSdata}\OperatorTok{$}\NormalTok{state}
\NormalTok{LAGOSnutrient <-}\StringTok{ }\NormalTok{LAGOSdata}\OperatorTok{$}\NormalTok{epi_nutr}
\NormalTok{LAGOSlimno <-}\StringTok{ }\NormalTok{LAGOSdata}\OperatorTok{$}\NormalTok{lakes_limno}


\NormalTok{LAGOScombined <-}\StringTok{ }
\StringTok{  }\KeywordTok{left_join}\NormalTok{(LAGOSnutrient, LAGOSlocus) }\OperatorTok
\StringTok{  }\KeywordTok{left_join}\NormalTok{(., LAGOSlimno) }\OperatorTok
\StringTok{  }\KeywordTok{left_join}\NormalTok{(., LAGOSstate) }\OperatorTok
\StringTok{  }\KeywordTok{filter}\NormalTok{(}\OperatorTok{!}\KeywordTok{is.na}\NormalTok{(state)) }\OperatorTok
\StringTok{  }\KeywordTok{select}\NormalTok{(lagoslakeid, sampledate, secchi, lake_area_ha, maxdepth, nhd_lat, nhd_long, state)}
\end{Highlighting}
\end{Shaded}

\begin{verbatim}
## Joining, by = "lagoslakeid"
\end{verbatim}

\begin{verbatim}
## Joining, by = c("lagoslakeid", "nhdid", "nhd_lat", "nhd_long")
\end{verbatim}

\begin{verbatim}
## Joining, by = "state_zoneid"
\end{verbatim}

\begin{Shaded}
\begin{Highlighting}[]
\NormalTok{secchi.maine <-}\StringTok{ }\NormalTok{LAGOScombined }\OperatorTok
\StringTok{  }\KeywordTok{filter}\NormalTok{(state }\OperatorTok{==}\StringTok{ "ME"}\NormalTok{) }\OperatorTok
\StringTok{  }\KeywordTok{group_by}\NormalTok{(lagoslakeid) }\OperatorTok
\StringTok{  }\KeywordTok{summarise}\NormalTok{(}\DataTypeTok{secchi.mean =} \KeywordTok{mean}\NormalTok{(secchi),}
            \DataTypeTok{area =} \KeywordTok{mean}\NormalTok{(lake_area_ha),}
            \DataTypeTok{depth =} \KeywordTok{mean}\NormalTok{(maxdepth),}
            \DataTypeTok{lat =} \KeywordTok{mean}\NormalTok{(nhd_lat), }
            \DataTypeTok{long =} \KeywordTok{mean}\NormalTok{(nhd_long)) }\OperatorTok\StringTok{ }
\StringTok{  }\KeywordTok{drop_na}\NormalTok{()}


\NormalTok{maine.subset <-}\StringTok{ }\KeywordTok{filter}\NormalTok{(states, ID }\OperatorTok\StringTok{ }
\StringTok{                          }\KeywordTok{c}\NormalTok{(}\StringTok{"maine"}\NormalTok{))}


\NormalTok{maine.spatial <-}\StringTok{ }\KeywordTok{st_as_sf}\NormalTok{(secchi.maine, }\DataTypeTok{coords =} \KeywordTok{c}\NormalTok{(}\StringTok{"long"}\NormalTok{, }\StringTok{"lat"}\NormalTok{), }\DataTypeTok{crs =} \DecValTok{4326}\NormalTok{)}
\end{Highlighting}
\end{Shaded}

\begin{enumerate}
\def\labelenumi{\arabic{enumi}.}
\setcounter{enumi}{5}
\tightlist
\item
  Create a plot of mean secchi depth for lakes in Maine, with mean
  secchi depth designated as color and the lake area as the size of the
  dot. Remember that you are using size in the aesthetics and should
  remove the size = 1 from the other part of the code. Adjust the
  transparency of points as needed.
\end{enumerate}

\begin{Shaded}
\begin{Highlighting}[]
\NormalTok{Maineplot.area <-}\StringTok{ }\KeywordTok{ggplot}\NormalTok{() }\OperatorTok{+}
\StringTok{  }\KeywordTok{geom_sf}\NormalTok{(}\DataTypeTok{data =}\NormalTok{ maine.subset, }\DataTypeTok{fill =} \StringTok{"white"}\NormalTok{) }\OperatorTok{+}
\StringTok{  }\KeywordTok{geom_sf}\NormalTok{(}\DataTypeTok{data =}\NormalTok{ maine.spatial, }\KeywordTok{aes}\NormalTok{(}\DataTypeTok{color =}\NormalTok{ secchi.mean, }\DataTypeTok{size =}\NormalTok{ area), }
          \DataTypeTok{alpha =} \FloatTok{0.5}\NormalTok{) }\OperatorTok{+}
\StringTok{  }\KeywordTok{scale_color_viridis_c}\NormalTok{() }\OperatorTok{+}
\StringTok{  }\KeywordTok{labs}\NormalTok{(}\DataTypeTok{color =} \StringTok{"Average Secchi Depth (m)"}\NormalTok{) }\OperatorTok{+}
\StringTok{  }\KeywordTok{theme}\NormalTok{(}\DataTypeTok{legend.position =} \StringTok{"top"}\NormalTok{)}
\KeywordTok{print}\NormalTok{(Maineplot.area)}
\end{Highlighting}
\end{Shaded}

\includegraphics{A08_Roth_files/figure-latex/unnamed-chunk-2-1.pdf}

\begin{enumerate}
\def\labelenumi{\arabic{enumi}.}
\setcounter{enumi}{6}
\tightlist
\item
  Create a second plot, but this time use maximum depth of the lake as
  the size of the dot.
\end{enumerate}

\begin{Shaded}
\begin{Highlighting}[]
\NormalTok{Maineplot.depth <-}\StringTok{ }\KeywordTok{ggplot}\NormalTok{() }\OperatorTok{+}
\StringTok{  }\KeywordTok{geom_sf}\NormalTok{(}\DataTypeTok{data =}\NormalTok{ maine.subset, }\DataTypeTok{fill =} \StringTok{"white"}\NormalTok{) }\OperatorTok{+}
\StringTok{  }\KeywordTok{geom_sf}\NormalTok{(}\DataTypeTok{data =}\NormalTok{ maine.spatial, }\KeywordTok{aes}\NormalTok{(}\DataTypeTok{color =}\NormalTok{ secchi.mean, }\DataTypeTok{size =}\NormalTok{ depth), }
          \DataTypeTok{alpha =} \FloatTok{0.5}\NormalTok{) }\OperatorTok{+}
\StringTok{  }\KeywordTok{scale_color_viridis_c}\NormalTok{() }\OperatorTok{+}
\StringTok{  }\KeywordTok{labs}\NormalTok{(}\DataTypeTok{color =} \StringTok{"Average Secchi Depth (m)"}\NormalTok{) }\OperatorTok{+}
\StringTok{  }\KeywordTok{theme}\NormalTok{(}\DataTypeTok{legend.position =} \StringTok{"top"}\NormalTok{)}
\KeywordTok{print}\NormalTok{(Maineplot.depth)}
\end{Highlighting}
\end{Shaded}

\includegraphics{A08_Roth_files/figure-latex/unnamed-chunk-3-1.pdf}

\begin{enumerate}
\def\labelenumi{\arabic{enumi}.}
\setcounter{enumi}{7}
\tightlist
\item
  Plot these maps in the same plot with the \texttt{plot\_grid}
  function. Don't worry about adjusting the legends (if you have extra
  time this would be a good bonus task).
\end{enumerate}

\begin{Shaded}
\begin{Highlighting}[]
\KeywordTok{plot_grid}\NormalTok{(Maineplot.area, Maineplot.depth)}
\end{Highlighting}
\end{Shaded}

\includegraphics{A08_Roth_files/figure-latex/unnamed-chunk-4-1.pdf}

What would you change about the legend to make it a more effective
visualization?

\begin{quote}
I would make the same colors correspond with the same secchi depths, and
I would have the shapes of the circles represented in the legend as a
guide for what areas/lake depths correspond with which size circle on
the graph.
\end{quote}

\begin{enumerate}
\def\labelenumi{\arabic{enumi}.}
\setcounter{enumi}{8}
\tightlist
\item
  What relationships do you see between secchi depth, lake area, and
  lake depth? Which of the two lake variables seems to be a stronger
  determinant of secchi depth? (make a scatterplot and run a regression
  to test this)
\end{enumerate}

\emph{Note: consider log-transforming a predictor variable if
appropriate}

\begin{Shaded}
\begin{Highlighting}[]
\KeywordTok{qqnorm}\NormalTok{(secchi.maine}\OperatorTok{$}\NormalTok{area)}
\KeywordTok{qqline}\NormalTok{(secchi.maine}\OperatorTok{$}\NormalTok{area)}
\end{Highlighting}
\end{Shaded}

\includegraphics{A08_Roth_files/figure-latex/unnamed-chunk-5-1.pdf}

\begin{Shaded}
\begin{Highlighting}[]
\KeywordTok{shapiro.test}\NormalTok{(secchi.maine}\OperatorTok{$}\NormalTok{area)}
\end{Highlighting}
\end{Shaded}

\begin{verbatim}
## 
##  Shapiro-Wilk normality test
## 
## data:  secchi.maine$area
## W = 0.2577, p-value < 2.2e-16
\end{verbatim}

\begin{Shaded}
\begin{Highlighting}[]
\KeywordTok{qqnorm}\NormalTok{(}\KeywordTok{log}\NormalTok{(secchi.maine}\OperatorTok{$}\NormalTok{area))}
\KeywordTok{qqline}\NormalTok{(}\KeywordTok{log}\NormalTok{(secchi.maine}\OperatorTok{$}\NormalTok{area))}
\end{Highlighting}
\end{Shaded}

\includegraphics{A08_Roth_files/figure-latex/unnamed-chunk-5-2.pdf}

\begin{Shaded}
\begin{Highlighting}[]
\KeywordTok{shapiro.test}\NormalTok{(}\KeywordTok{log}\NormalTok{(secchi.maine}\OperatorTok{$}\NormalTok{area))}
\end{Highlighting}
\end{Shaded}

\begin{verbatim}
## 
##  Shapiro-Wilk normality test
## 
## data:  log(secchi.maine$area)
## W = 0.9831, p-value = 5.447e-06
\end{verbatim}

\begin{Shaded}
\begin{Highlighting}[]
\KeywordTok{qqnorm}\NormalTok{(secchi.maine}\OperatorTok{$}\NormalTok{depth)}
\KeywordTok{qqline}\NormalTok{(secchi.maine}\OperatorTok{$}\NormalTok{depth)}
\end{Highlighting}
\end{Shaded}

\includegraphics{A08_Roth_files/figure-latex/unnamed-chunk-5-3.pdf}

\begin{Shaded}
\begin{Highlighting}[]
\KeywordTok{shapiro.test}\NormalTok{(secchi.maine}\OperatorTok{$}\NormalTok{depth)}
\end{Highlighting}
\end{Shaded}

\begin{verbatim}
## 
##  Shapiro-Wilk normality test
## 
## data:  secchi.maine$depth
## W = 0.83016, p-value < 2.2e-16
\end{verbatim}

\begin{Shaded}
\begin{Highlighting}[]
\KeywordTok{qqnorm}\NormalTok{(}\KeywordTok{log}\NormalTok{(secchi.maine}\OperatorTok{$}\NormalTok{depth))}
\KeywordTok{qqline}\NormalTok{(}\KeywordTok{log}\NormalTok{(secchi.maine}\OperatorTok{$}\NormalTok{depth))}
\end{Highlighting}
\end{Shaded}

\includegraphics{A08_Roth_files/figure-latex/unnamed-chunk-5-4.pdf}

\begin{Shaded}
\begin{Highlighting}[]
\KeywordTok{shapiro.test}\NormalTok{(}\KeywordTok{log}\NormalTok{(secchi.maine}\OperatorTok{$}\NormalTok{depth))}
\end{Highlighting}
\end{Shaded}

\begin{verbatim}
## 
##  Shapiro-Wilk normality test
## 
## data:  log(secchi.maine$depth)
## W = 0.99505, p-value = 0.07565
\end{verbatim}

\begin{Shaded}
\begin{Highlighting}[]
\NormalTok{maine.test <-}\StringTok{ }\KeywordTok{lm}\NormalTok{(secchi.mean }\OperatorTok{~}\StringTok{ }\KeywordTok{log}\NormalTok{(area) }\OperatorTok{+}\StringTok{ }\KeywordTok{log}\NormalTok{(depth), secchi.maine)}
\KeywordTok{summary}\NormalTok{(maine.test)}
\end{Highlighting}
\end{Shaded}

\begin{verbatim}
## 
## Call:
## lm(formula = secchi.mean ~ log(area) + log(depth), data = secchi.maine)
## 
## Residuals:
##     Min      1Q  Median      3Q     Max 
## -3.8424 -0.9754 -0.0069  0.8011  9.4708 
## 
## Coefficients:
##             Estimate Std. Error t value Pr(>|t|)    
## (Intercept)  0.53440    0.22100   2.418  0.01593 *  
## log(area)   -0.14439    0.04864  -2.968  0.00312 ** 
## log(depth)   2.06428    0.09851  20.954  < 2e-16 ***
## ---
## Signif. codes:  0 '***' 0.001 '**' 0.01 '*' 0.05 '.' 0.1 ' ' 1
## 
## Residual standard error: 1.572 on 546 degrees of freedom
## Multiple R-squared:  0.4831, Adjusted R-squared:  0.4812 
## F-statistic: 255.1 on 2 and 546 DF,  p-value: < 2.2e-16
\end{verbatim}

\begin{quote}
Lake area and lake depth both significantly determine secchi depth in
Maine. Secchi depth decreases with lake area and increases with lake
depth. Lake depth has a stronger influence on secchi depth (p
\textless{} 2e-16) than lake area (p = 0.00312).
\end{quote}

\hypertarget{mapping-water-features-and-watershed-boundaries}{%
\subsection{Mapping water features and watershed
boundaries}\label{mapping-water-features-and-watershed-boundaries}}

\begin{enumerate}
\def\labelenumi{\arabic{enumi}.}
\setcounter{enumi}{9}
\tightlist
\item
  Wrangle the USA rivers and HUC6 watershed boundaries dataset so that
  they include only the features present in Florida (FL). Adjust the
  coordinate reference systems if necessary to ensure they use the same
  projection.
\end{enumerate}

\begin{Shaded}
\begin{Highlighting}[]
\KeywordTok{summary}\NormalTok{(HUC6}\OperatorTok{$}\NormalTok{States)}
\end{Highlighting}
\end{Shaded}

\begin{verbatim}
##       AL    AL,FL AL,FL,GA AL,GA,TN AL,LA,MS    AL,MS       FL    FL,GA 
##        1        3        1        1        1        2        6        4 
##       GA GA,NC,SC    LA,MS       NC    NC,SC NC,SC,VA    NC,VA       SC 
##        2        1        1        4        2        1        2        1
\end{verbatim}

\begin{Shaded}
\begin{Highlighting}[]
\NormalTok{HUC6.FL <-}\StringTok{ }\NormalTok{HUC6 }\OperatorTok
\StringTok{  }\KeywordTok{filter}\NormalTok{(States }\OperatorTok\StringTok{ }\KeywordTok{c}\NormalTok{(}\StringTok{"AL,FL"}\NormalTok{, }\StringTok{"AL,FL,GA"}\NormalTok{, }\StringTok{"FL"}\NormalTok{, }\StringTok{"FL,GA"}\NormalTok{))}


\NormalTok{waterfeatures <-}\StringTok{ }\KeywordTok{filter}\NormalTok{(waterfeatures, STATE }\OperatorTok{==}\StringTok{ "FL"}\NormalTok{)}


\NormalTok{waterfeatures <-}\StringTok{ }\KeywordTok{filter}\NormalTok{(waterfeatures, FEATURE }\OperatorTok{!=}\StringTok{ "Apparent Limit"} \OperatorTok{&}\StringTok{ }\NormalTok{FEATURE }\OperatorTok{!=}\StringTok{ "Closure Line"}\NormalTok{)}
\end{Highlighting}
\end{Shaded}

\begin{enumerate}
\def\labelenumi{\arabic{enumi}.}
\setcounter{enumi}{10}
\tightlist
\item
  Create a map of watershed boundaries in Florida, with the layer of
  water features on top. Color the watersheds gray (make sure the lines
  separating watersheds are still visible) and color the water features
  by type.
\end{enumerate}

\begin{Shaded}
\begin{Highlighting}[]
\KeywordTok{st_crs}\NormalTok{(waterfeatures)}
\end{Highlighting}
\end{Shaded}

\begin{verbatim}
## Coordinate Reference System: NA
\end{verbatim}

\begin{Shaded}
\begin{Highlighting}[]
\KeywordTok{st_crs}\NormalTok{(HUC6.FL)}
\end{Highlighting}
\end{Shaded}

\begin{verbatim}
## Coordinate Reference System:
##   EPSG: 4269 
##   proj4string: "+proj=longlat +ellps=GRS80 +towgs84=0,0,0,0,0,0,0 +no_defs"
\end{verbatim}

\begin{Shaded}
\begin{Highlighting}[]
\NormalTok{waterfeatures <-}\StringTok{ }\KeywordTok{st_set_crs}\NormalTok{(waterfeatures, }\DecValTok{4269}\NormalTok{) }
\KeywordTok{st_crs}\NormalTok{(waterfeatures)}
\end{Highlighting}
\end{Shaded}

\begin{verbatim}
## Coordinate Reference System:
##   EPSG: 4269 
##   proj4string: "+proj=longlat +ellps=GRS80 +towgs84=0,0,0,0,0,0,0 +no_defs"
\end{verbatim}

\begin{Shaded}
\begin{Highlighting}[]
\NormalTok{FLlayers <-}\StringTok{ }\KeywordTok{ggplot}\NormalTok{() }\OperatorTok{+}
\StringTok{  }\KeywordTok{geom_sf}\NormalTok{(}\DataTypeTok{data =}\NormalTok{ HUC6.FL, }\DataTypeTok{color =} \StringTok{"gray"}\NormalTok{, }\DataTypeTok{alpha =} \FloatTok{0.5}\NormalTok{) }\OperatorTok{+}\StringTok{ }
\StringTok{  }\KeywordTok{geom_sf}\NormalTok{(}\DataTypeTok{data =}\NormalTok{ waterfeatures, }\KeywordTok{aes}\NormalTok{(}\DataTypeTok{color =}\NormalTok{ FEATURE)) }\OperatorTok{+}
\StringTok{  }\KeywordTok{scale_color_brewer}\NormalTok{(}\DataTypeTok{palette =} \StringTok{"Paired"}\NormalTok{)}
\KeywordTok{print}\NormalTok{(FLlayers)}
\end{Highlighting}
\end{Shaded}

\includegraphics{A08_Roth_files/figure-latex/unnamed-chunk-7-1.pdf}

\begin{enumerate}
\def\labelenumi{\arabic{enumi}.}
\setcounter{enumi}{11}
\tightlist
\item
  What are the dominant water features in Florida? How does this
  distribution differ (or not) compared to North Carolina?
\end{enumerate}

\begin{quote}
The dominant water features in Florida are streams, shorelines, and
canals. North Carolina also has a large number of streams and rivers,
but has a smaller amount of shoreline and fewer canals.
\end{quote}

\hypertarget{reflection}{%
\subsection{Reflection}\label{reflection}}

\begin{enumerate}
\def\labelenumi{\arabic{enumi}.}
\setcounter{enumi}{12}
\tightlist
\item
  What are 2-3 conclusions or summary points about mapping you learned
  through your analysis?
\end{enumerate}

\begin{quote}
I learned that secchi depth is more dependent on lake depth than lake
area in Maine and the state of Florida's waterways are primarily
streams, shoreline, and canals.
\end{quote}

\begin{enumerate}
\def\labelenumi{\arabic{enumi}.}
\setcounter{enumi}{13}
\tightlist
\item
  What data, visualizations, and/or models supported your conclusions
  from 13?
\end{enumerate}

\begin{quote}
The linear model helped me determine the first conclusion, and the
spatial visualization helped me reach the second conclusion.
\end{quote}

\begin{enumerate}
\def\labelenumi{\arabic{enumi}.}
\setcounter{enumi}{14}
\tightlist
\item
  Did hands-on data analysis impact your learning about mapping relative
  to a theory-based lesson? If so, how?
\end{enumerate}

\begin{quote}
Hands on analysis always helps be learn better than a theory-based
lesson because I am a visual learner.
\end{quote}

\begin{enumerate}
\def\labelenumi{\arabic{enumi}.}
\setcounter{enumi}{15}
\tightlist
\item
  How did the real-world data compare with your expectations from
  theory?
\end{enumerate}

\begin{quote}
I was not expecting as many canals in Florida as there are.
\end{quote}


\end{document}
